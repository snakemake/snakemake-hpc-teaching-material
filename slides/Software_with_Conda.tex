\section{Conda on Mogon}

%%%%%%%%%%%%%%%%%%%%%%%%%%%%%%%%%%%%%%%%%%%%%%%%%%%%%%%%%%%%%%%%%%%%%%%%%%%%%%%%
\begin{frame}
    \frametitle{Outline}
    \begin{columns}[t]
        \begin{column}{.5\textwidth}
            \tableofcontents[sections={1-9},currentsection]
        \end{column}
        \begin{column}{.5\textwidth}
            \tableofcontents[sections={10-18},currentsection]
        \end{column}
    \end{columns}
\end{frame}

%%%%%%%%%%%%%%%%%%%%%%%%%%%%%%%%%%%%%%%%%%%%%%%%%%%%%%%%%%%%%%%%%%%%%%%%%%%%%%%% 
\begin{frame}<handout:0> 
  \frametitle{Your Work Environment with Conda}
  \begin{columns}
    \begin{column}{0.5\textwidth}\centering
      \includegraphics[width=0.8\textwidth]{environment/environment.png}
    \end{column}
    \begin{column}{0.5\textwidth}\centering
      \includegraphics[width=0.8\textwidth]{logos/Conda_logo.png}   
    \end{column}
  \end{columns}
\end{frame}

%%%%%%%%%%%%%%%%%%%%%%%%%%%%%%%%%%%%%%%%%%%%%%%%%%%%%%%%%%%%%%%%%%%%%%%%%%%%%%%% 
\begin{frame}
  \frametitle{Disclaimer}
  \warning{
    \begin{itemize}
     \item The following description does not reflect the HPC-groups opinion on Conda.
     \item Some of the following slides contain information which is specific to \mogon, e.\,g. path names. 
     \item Some of the following slides contain valuable information in any case.
    \end{itemize}
  }
\end{frame}

%%%%%%%%%%%%%%%%%%%%%%%%%%%%%%%%%%%%%%%%%%%%%%%%%%%%%%%%%%%%%%%%%%%%%%%%%%%%%%%% 
\begin{frame}[fragile]
  \frametitle{\Interlude{Getting Your Quota Limits}}
  \begin{lstlisting}[language=Bash, style=Shell]
$ mmlsquota --block-size auto -u $USER fs1:home | grep -v 'limits'
  \end{lstlisting}
  reports your number of files.
  \begin{alertblock}{Quota on \mogon}
   Currently your home directory has the following limits:
   \begin{itemize}
    \item 150.000 files and
    \item a quota of \SI{50}{\gibi\byte}
   \end{itemize}
   \pause
   \hint{By asking for our quota first, we learn how many file we will produce by installing Conda!}
  \end{alertblock}
\end{frame}

%%%%%%%%%%%%%%%%%%%%%%%%%%%%%%%%%%%%%%%%%%%%%%%%%%%%%%%%%%%%%%%%%%%%%%%%%%%%%%%% 
\begin{frame}[fragile]
  \frametitle{Getting Conda}
  On Linux, you can retrieve any given URL from the command line with ``\texttt{wget}''. 
  \question{We are NOT getting our Conda via a module. Why?}
  \pause
  \docs{Because almost ever the module will be outdated and the prefix-issue is simpler to resolve without the module.}
\end{frame}

%%%%%%%%%%%%%%%%%%%%%%%%%%%%%%%%%%%%%%%%%%%%%%%%%%%%%%%%%%%%%%%%%%%%%%%%%%%%%%%% 
\begin{frame}[fragile]
  \frametitle{Installing Conda}
  You could run
  \begin{lstlisting}[language=Bash, style=Shell, basicstyle=\small,breaklines=true ]
$ wget https://repo.anaconda.com/miniconda/Miniconda3-latest-Linux-x86_64.sh
  \end{lstlisting}
  \hint{\footnotesize  URL is from \url{https://docs.conda.io/en/latest/miniconda.html}.\newline However, we are going to use our copied script.}
  \pause
  \docs{\footnotesize Using the ``\emph{latest}'' version of the script, we ensure not to rely on a specific version. It is always possible to install specific software releases with the current version of Conda, but we will not be required to update conda immediately after installation.}
  \hint{Instead of downloading, we will work through this together on the slides to come.}
 \end{frame} 
 
%%%%%%%%%%%%%%%%%%%%%%%%%%%%%%%%%%%%%%%%%%%%%%%%%%%%%%%%%%%%%%%%%%%%%%%%%%%%%%%% 
\begin{frame}[fragile]
  \frametitle{Installing Conda - II}
  Please run
  \begin{lstlisting}[language=Bash, style=Shell]
$ bash ./setup/install_mamba.sh
  \end{lstlisting}
  We will work through the questions together, please be patient!
\end{frame}

% curl -L https://github.com/conda-forge/miniforge/releases/latest/download/Mambaforge-Linux-x86_64.sh -o Mambaforge-Linux-x86_64.sh

%%%%%%%%%%%%%%%%%%%%%%%%%%%%%%%%%%%%%%%%%%%%%%%%%%%%%%%%%%%%%%%%%%%%%%%%%%%%%%%% 
\begin{frame}[fragile]
  \frametitle{Installing Conda - III}
  \begin{itemize}[<+->]
   \item You need to confirm the license. Scroll with ``blank'' and confirm with ``\verb+yes+''.
   \item Next, you need to set the prefix. For the course you can install Conda in your HOME, else, agree in your group(s) on a path in your project and use the project directory, e.g. 
   \begin{lstlisting}[language=Bash, style=Shell, breaklines=true ]
[/home/<username>/mambaforge] >>> /lustre/project/<myproject>/mambaforge
   \end{lstlisting}
   \item Finally, you will be asked whether everything is. Please confirm with ``\verb+yes+''.
  \end{itemize}
\end{frame}

%%%%%%%%%%%%%%%%%%%%%%%%%%%%%%%%%%%%%%%%%%%%%%%%%%%%%%%%%%%%%%%%%%%%%%%%%%%%%%%% 
\begin{frame}[fragile]
  \frametitle{Installing Conda - IV}
  \footnotesize
  \begin{columns}[t]
    \begin{column}{0.5\textwidth}
       You now have a section like this in your ``\texttt{\textasciitilde/.bashrc}'':
       \begin{lstlisting}[language=Bash, style=Shell, basicstyle=\tiny, breaklines=true]
# >>> conda initialize >>>
# !! Contents within this block are managed by 'conda init' !!
__conda_setup="$('<prefix>/bin/conda' 'shell.bash' 'hook' 2> /dev/null)"
if [ $? -eq 0 ]; then
    eval "$__conda_setup"
else
    if [ -f "<prefix>/etc/profile.d/conda.sh" ]; then
        . "<prefix>/etc/profile.d/conda.sh"
    else
        export PATH="<prefix>/bin:$PATH"
    fi
fi
unset __conda_setup
# <<< conda initialize <<<
      \end{lstlisting}
      \bcattention \emph{Every} time you log-in this will be executed. Also, here, ``\texttt{<prefix>}'' denotes \emph{your} prefix.
    \end{column}
    \begin{column}{0.5\textwidth}
       \pause
       If you like, you can put this part in a function, to re-gain manual control:
       \begin{lstlisting}[language=Bash, style=Shell, basicstyle=\tiny, breaklines=true]
@function conda_initialize {@
# >>> conda initialize >>>
# !! Contents within this block are managed by 'conda init' !!
__conda_setup="$('<prefix>/bin/conda' 'shell.bash' 'hook' 2> /dev/null)"
if [ $? -eq 0 ]; then
    eval "$__conda_setup"
else
    if [ -f "<prefix>/etc/profile.d/conda.sh" ]; then
        . "<prefix>/etc/profile.d/conda.sh"
    else
        export PATH="<prefix>/bin:$PATH"
    fi
fi
unset __conda_setup
# <<< conda initialize <<<
@}@
      \end{lstlisting}
    \end{column}
  \end{columns}

   %
\end{frame}

%%%%%%%%%%%%%%%%%%%%%%%%%%%%%%%%%%%%%%%%%%%%%%%%%%%%%%%%%%%%%%%%%%%%%%%%%%%%%%%% 
\begin{frame}[fragile]
  \frametitle{Initializing Conda}
  To initialize Conda, simply run
  \begin{lstlisting}[language=Bash, style=Shell]
$ bash
  \end{lstlisting}
  or
  \begin{lstlisting}[language=Bash, style=Shell]
$ source ~/.bashrc
$ conda_initialize # if you have this function
  \end{lstlisting}
  Then your command line should look like:
  \begin{lstlisting}[language=Bash, style=Shell]
(base) ... $
  \end{lstlisting}
  , where \verb+(base)+ is the ``base'' environment of Conda. This is your starting environment.
\end{frame}

%%%%%%%%%%%%%%%%%%%%%%%%%%%%%%%%%%%%%%%%%%%%%%%%%%%%%%%%%%%%%%%%%%%%%%%%%%%%%%%% 
\begin{frame}[fragile]
  \frametitle{Searching Software with Conda}
  First you might have to look for software. This is done with
  \begin{lstlisting}[language=Bash, style=Shell]
$ conda search <softwarename>
  \end{lstlisting}
  \pause
  \task{Try this with a software which comes to mind.}
  \pause
  This will list packages with channel and version information, e.\,g.
  \begin{lstlisting}[language=Bash, style=Shell, basicstyle=\tiny]
$ conda search minimap
Loading channels: done
# Name                       Version           Build  Channel             
minimap                     0.2_r124               0  bioconda            
minimap                     0.2_r124      h5bf99c6_4  bioconda
....
  \end{lstlisting}
\end{frame}

%%%%%%%%%%%%%%%%%%%%%%%%%%%%%%%%%%%%%%%%%%%%%%%%%%%%%%%%%%%%%%%%%%%%%%%%%%%%%%%% 
\begin{frame}[fragile]
  \frametitle{Reducing Search Overhead - the \texttt{.condarc}-File}
  On \mogon{} the number of Conda channels is reduced by the whitelisting. Nevertheless, it helps to have a resource file, with a number of definitions, \emph{before} starting:
  \begin{lstlisting}[language=Bash, style=Shell, basicstyle=\tiny]
$ cat .condarc
create_default_packages:
  - setuptools
channels:
  - conda-forge
  - bioconda
  - defaults
  - r
proxy_servers:
  http: http://webproxy.zdv.uni-mainz.de:8888
ssl_verify: false
auto_update_conda: false
always_yes: true # avoid confirmation(s)
  \end{lstlisting}
  To obtain the same resource file, run:
  \begin{lstlisting}[language=Bash, style=Shell, basicstyle=\footnotesize]
$ cp ~/conda/condarc ~/.condarc
  \end{lstlisting}
\end{frame}

%%%%%%%%%%%%%%%%%%%%%%%%%%%%%%%%%%%%%%%%%%%%%%%%%%%%%%%%%%%%%%%%%%%%%%%%%%%%%%%% 
\begin{frame}[fragile]
  \frametitle{Installing Software \emph{with} Conda}
  The first software we want to install (in our ``base'' environment) is ``\texttt{mamba}''.  
  \docs{``\texttt{mamba}'' is a drop-in for \texttt{``conda''}, written in \CC. This means all commands are exactly the same as with \texttt{``conda''} (with some exceptions. It will be replaced with ``\texttt{pixi}''.}
  Please run
  \begin{lstlisting}[language=Bash, style=Shell]
$ conda install mamba
  \end{lstlisting}
\end{frame}

%%%%%%%%%%%%%%%%%%%%%%%%%%%%%%%%%%%%%%%%%%%%%%%%%%%%%%%%%%%%%%%%%%%%%%%%%%%%%%%% 
\begin{frame}[fragile]
  \frametitle{Installing Software \emph{with} Conda - II}
  As we will be using our workflow system in \emph{every} environment, we can as well use it in the ``\texttt{base}'' environment:
  \begin{lstlisting}[language=Bash, style=Shell]
$ mamba install snakemake
  \end{lstlisting}
\end{frame}

%%%%%%%%%%%%%%%%%%%%%%%%%%%%%%%%%%%%%%%%%%%%%%%%%%%%%%%%%%%%%%%%%%%%%%%%%%%%%%%% 
\begin{frame}[fragile]
  \frametitle{Installing Software \emph{with} Conda - Using Environments}
  \hint{It is a good habit to have
        \begin{itemize}
          \item \emph{an} environment per workflow
          \item the environment named as the workflow
          \item this way, we have a bundle of tools, activate the environment for it
          \item \texttt{snakemake} workflows will install the tools you need for a particular workflow - only \emph{if} these tools are still missing
         \end{itemize}
        }
  Please run:
  \begin{lstlisting}[language=Bash, style=Shell]
$ conda create -n snakemake-tutorial
  \end{lstlisting}
  \pause

  To list the available environments we can run
  \begin{lstlisting}[language=Bash, style=Shell]
$ conda info --envs
  \end{lstlisting}
\end{frame}

%%%%%%%%%%%%%%%%%%%%%%%%%%%%%%%%%%%%%%%%%%%%%%%%%%%%%%%%%%%%%%%%%%%%%%%%%%%%%%%% 
\begin{frame}[fragile]
  \frametitle{Installing Software \emph{with} Conda - Using Environments II}
  Using our new environment we can activate it:
  \begin{lstlisting}[language=Bash, style=Shell]
$ conda activate snakemake-tutorial
  \end{lstlisting}
  and install ``\texttt{snakemake}'' by running
  \begin{lstlisting}[language=Bash, style=Shell]
$ mamba install snakemake
  \end{lstlisting}
  Now, we are all set. ``\texttt{snakemake}'' is able to install the requested software, for a well-maintained workflow.
\end{frame}

%%%%%%%%%%%%%%%%%%%%%%%%%%%%%%%%%%%%%%%%%%%%%%%%%%%%%%%%%%%%%%%%%%%%%%%%%%%%%%%% 
\begin{frame}[fragile]
  \frametitle{Conda - a workflow summary}
  To handle a given workflow, do
  \begin{lstlisting}[language=Bash, style=Shell]
$ conda_initialize
$ conda create -n <workflow_name> # if not present: create environment
$ conda activate <workflow_name>
$ mamba install snakemake
  \end{lstlisting}
\end{frame}

