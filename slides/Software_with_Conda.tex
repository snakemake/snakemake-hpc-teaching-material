\section{Conda on Mogon}

%%%%%%%%%%%%%%%%%%%%%%%%%%%%%%%%%%%%%%%%%%%%%%%%%%%%%%%%%%%%%%%%%%%%%%%%%%%%%%%%
\begin{frame}
    \frametitle{Outline}
    \begin{columns}[t]
        \begin{column}{.5\textwidth}
            \tableofcontents[sections={1-9},currentsection]
        \end{column}
        \begin{column}{.5\textwidth}
            \tableofcontents[sections={10-18},currentsection]
        \end{column}
    \end{columns}
\end{frame}

%%%%%%%%%%%%%%%%%%%%%%%%%%%%%%%%%%%%%%%%%%%%%%%%%%%%%%%%%%%%%%%%%%%%%%%%%%%%%%%% 
\begin{frame}<handout:0> 
  \frametitle{Your Work Environment with Conda}
  \begin{columns}
    \begin{column}{0.5\textwidth}\centering
      \includegraphics[width=0.8\textwidth]{environment/environment.png}
    \end{column}
    \begin{column}{0.5\textwidth}\centering
      \includegraphics[width=0.8\textwidth]{logos/Conda_logo.png}   
    \end{column}
  \end{columns}
\end{frame}

%%%%%%%%%%%%%%%%%%%%%%%%%%%%%%%%%%%%%%%%%%%%%%%%%%%%%%%%%%%%%%%%%%%%%%%%%%%%%%%% 
\begin{frame}
  \frametitle{Disclaimer}
  \warning{
    \begin{itemize}
     \item The following description does not reflect the HPC-groups opinion on Conda.
     \item Some of the following slides contain information which is specific to \mogon, e.\,g. path names. 
     \item Some of the following slides contain valuable information in any case.
    \end{itemize}
  }
\end{frame}


%%%%%%%%%%%%%%%%%%%%%%%%%%%%%%%%%%%%%%%%%%%%%%%%%%%%%%%%%%%%%%%%%%%%%%%%%%%%%%%% 
\begin{frame}[fragile]
  \frametitle{Getting Conda}
  On Linux, you can retrieve any given URL from the command line with ``\texttt{wget}''. 
  \question{We are NOT getting our Conda via a module. Why?}
  \pause
  \docs{Because almost ever the module will be outdated and the prefix-issue is simpler to resolve without the module.}
\end{frame}

%%%%%%%%%%%%%%%%%%%%%%%%%%%%%%%%%%%%%%%%%%%%%%%%%%%%%%%%%%%%%%%%%%%%%%%%%%%%%%%% 
\begin{frame}[fragile]
  \frametitle{Installing Conda}
  You could run
  \begin{lstlisting}[language=Bash, style=Shell, basicstyle=\small,breaklines=true ]
$ wget https://repo.anaconda.com/miniconda/Miniconda3-latest-Linux-x86_64.sh
  \end{lstlisting}
  \hint{\footnotesize  URL is from \url{https://docs.conda.io/en/latest/miniconda.html}.\newline However, we are going to use our copied script.}
  \pause
  \docs{\footnotesize Using the ``\emph{latest}'' version of the script, we ensure not to rely on a specific version. It is always possible to install specific software releases with the current version of Conda, but we will not be required to update conda immediately after installation.}
  \hint{Instead of downloading, we will work through this together on the slides to come.}
 \end{frame} 
 
%%%%%%%%%%%%%%%%%%%%%%%%%%%%%%%%%%%%%%%%%%%%%%%%%%%%%%%%%%%%%%%%%%%%%%%%%%%%%%%% 
\begin{frame}[fragile]
  \frametitle{Installing Conda - II}
  Please run
  \begin{lstlisting}[language=Bash, style=Shell]
$ bash ./conda/Miniconda3-latest-Linux-x86_64.sh
  \end{lstlisting}
  We will work through the questions together, please be patient!
\end{frame}

%%%%%%%%%%%%%%%%%%%%%%%%%%%%%%%%%%%%%%%%%%%%%%%%%%%%%%%%%%%%%%%%%%%%%%%%%%%%%%%% 
\begin{frame}[fragile]
  \frametitle{Installing Conda - III}
  \begin{itemize}[<+->]
   \item You need to confirm the license. Scroll with ``blank'' and confirm with ``\verb+yes+''.
   \item Next, you need to set the prefix. For the course you can install Conda in your HOME, else, agree in your group(s) on a path in your project and use the project directory, e.g. 
   \begin{lstlisting}[language=Bash, style=Shell, breaklines=true ]
[/home/<username>/miniconda3] >>> /lustre/project/<myproject>/conda
   \end{lstlisting}
   \item Finally, you will be asked to finalize Miniconda. Please confirm with ``\verb+yes+''.
  \end{itemize}
\end{frame}

%%%%%%%%%%%%%%%%%%%%%%%%%%%%%%%%%%%%%%%%%%%%%%%%%%%%%%%%%%%%%%%%%%%%%%%%%%%%%%%% 
\begin{frame}[fragile]
  \frametitle{Installing Conda - IV}
  \footnotesize
  \begin{columns}[t]
    \begin{column}{0.5\textwidth}
       You now have a section like this in your ``\texttt{\textasciitilde/.bashrc}'':
       \begin{lstlisting}[language=Bash, style=Shell, basicstyle=\tiny, breaklines=true]
# >>> conda initialize >>>
# !! Contents within this block are managed by 'conda init' !!
__conda_setup="$('<prefix>/bin/conda' 'shell.bash' 'hook' 2> /dev/null)"
if [ $? -eq 0 ]; then
    eval "$__conda_setup"
else
    if [ -f "<prefix>/etc/profile.d/conda.sh" ]; then
        . "<prefix>/etc/profile.d/conda.sh"
    else
        export PATH="<prefix>/bin:$PATH"
    fi
fi
unset __conda_setup
# <<< conda initialize <<<
      \end{lstlisting}
      \bcattention \emph{Every} time you log-in this will be executed. Also, here, ``\texttt{<prefix>}'' denotes \emph{your} prefix.
    \end{column}
    \begin{column}{0.5\textwidth}
       \pause
       If you like, you can put this part in a function, to re-gain manual control:
       \begin{lstlisting}[language=Bash, style=Shell, basicstyle=\tiny, breaklines=true]
@function conda_initialize {@
# >>> conda initialize >>>
# !! Contents within this block are managed by 'conda init' !!
__conda_setup="$('<prefix>/bin/conda' 'shell.bash' 'hook' 2> /dev/null)"
if [ $? -eq 0 ]; then
    eval "$__conda_setup"
else
    if [ -f "<prefix>/etc/profile.d/conda.sh" ]; then
        . "<prefix>/etc/profile.d/conda.sh"
    else
        export PATH="<prefix>/bin:$PATH"
    fi
fi
unset __conda_setup
# <<< conda initialize <<<
@}@
      \end{lstlisting}
    \end{column}
  \end{columns}

   %
\end{frame}

%%%%%%%%%%%%%%%%%%%%%%%%%%%%%%%%%%%%%%%%%%%%%%%%%%%%%%%%%%%%%%%%%%%%%%%%%%%%%%%% 
\begin{frame}[fragile]
  \frametitle{Initializing Conda}
  To initialize Conda, simply run
  \begin{lstlisting}[language=Bash, style=Shell]
$ bash
  \end{lstlisting}
  or
  \begin{lstlisting}[language=Bash, style=Shell]
$ source ~/.bashrc
$ conda_initialize # if you have this function
  \end{lstlisting}
  Then your command line should look like:
  \begin{lstlisting}[language=Bash, style=Shell]
(base) ... $
  \end{lstlisting}
  , where \verb+(base)+ is the ``base'' environment of Conda. This is your starting environment.
\end{frame}

%%%%%%%%%%%%%%%%%%%%%%%%%%%%%%%%%%%%%%%%%%%%%%%%%%%%%%%%%%%%%%%%%%%%%%%%%%%%%%%% 
\begin{frame}[fragile]
  \frametitle{Searching Software with Conda}
  First you might have to look for software. This is done with
  \begin{lstlisting}[language=Bash, style=Shell]
$ conda search <softwarename>
  \end{lstlisting}
  \pause
  \task{Try this with a software which comes to mind.}
  \pause
  This will list packages with channel and version information, e.\,g.
  \begin{lstlisting}[language=Bash, style=Shell, basicstyle=\tiny]
$ conda search minimap
Loading channels: done
# Name                       Version           Build  Channel             
minimap                     0.2_r124               0  bioconda            
minimap                     0.2_r124      h5bf99c6_4  bioconda
....
  \end{lstlisting}
\end{frame}
