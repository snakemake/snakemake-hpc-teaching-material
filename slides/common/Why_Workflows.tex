%%%%%%%%%%%%%%%%%%%%%%%%%%%%%%%%%%%%%%%%%%%%%%%%%%%%%%%%%%%%%%%%%%%%%%%%%%%%%%%%
\section{Why Workflows}
{   
	\usebackgroundtemplate{
		\vbox to \paperheight{\vfil\hbox to \paperwidth{\hfil\includegraphics[height=.7\paperheight]{humor/DALLE_LEGO-scientist-thinking}\hfil}\vfil}
	}
	\frame{
		\frametitle{Why use Workflow Managers?}
		\begin{mdframed}[tikzsetting={draw=white,fill=white,fill opacity=0.8,
				line width=0pt},backgroundcolor=none,leftmargin=0,
			rightmargin=150,innertopmargin=4pt,roundcorner=10pt]
			\tableofcontents[currentsection,sections={1-4},hideothersubsections]
		\end{mdframed}
	    \vspace{12mm}\hfill{\tiny \lhref{https://zenodo.org/records/11147887}{from Ewa Bres \& Christian Bittner}}
	}
}

%%%%%%%%%%%%%%%%%%%%%%%%%%%%%%%%%%%%%%%%%%%%%%%%%%%%%%%%%%%%%%%%%%%%%%%%%%%%%%%%
\begin{frame}
  \frametitle{What is this about?}
   \begin{question}[Questions]
   	 \begin{itemize}
        \item I can code everything! Can I?
        \item What is the benefit of a workflow system?
        \item What distinguishes a workflow system from a ``pipeline''?
     \end{itemize}
   \end{question}
   \begin{docs}[Objectives]
   	  \begin{enumerate}
         \item Introducing workflow engines - particularly \Snakemake!
      \end{enumerate}
   \end{docs}
\end{frame}  

%%%%%%%%%%%%%%%%%%%%%%%%%%%%%%%%%%%%%%%%%%%%%%%%%%%%%%%%%%%%%%%%%%%%%%%%%%%%%%%%
\begin{frame}
  \frametitle{Data Analysis}
  \begin{onlyenv}<1| handout:0>
    \begin{tikzpicture}
      \path[use as bounding box] (0.7,0) rectangle (12,8);
      \node[inner sep=0pt] (analysis_1) at (5,6)
         {\includegraphics[width=0.7\textwidth]{Snakemake/analysis_1.png}};   
      \node at (7, 3.5) %[below=-0.4cm of analysis_1, xshift=2.7cm] at (current page.center)
         {\includegraphics[width=0.45\textwidth]{Snakemake/phd_left.png}};
      \node at (6, 1) {\begin{minipage}{0.75\textwidth}\footnotesize
                        Idea from the official \lhref{https://slides.com/johanneskoester/snakemake-tutorial}{\Snakemake} course (with permission), image from \lhref{https://phdcomics.com/comics.php}{PhD comics}.
                       \end{minipage}
      };
    \end{tikzpicture}    
  \end{onlyenv}
  
  \begin{onlyenv}<2| handout:1>
    \begin{tikzpicture}
      \path[use as bounding box] (0.7,0) rectangle (12,8);
      \node[inner sep=0pt] (analysis_full) at (5,6)
         {\includegraphics[width=0.7\textwidth]{Snakemake/analysis_full.png}};   
      \node at (7,3.5) % [below=-0.4cm of analysis_full, xshift=2.7cm]
         {\includegraphics[width=0.45\textwidth]{Snakemake/phd_full.png}};
      \node at (6, 1) {\begin{minipage}{0.75\textwidth}\footnotesize
                        Idea from the official \lhref{https://slides.com/johanneskoester/snakemake-tutorial}{\Snakemake} course (with permission), image from \lhref{https://phdcomics.com/comics.php}{PhD comics}.
                       \end{minipage}
      };
    \end{tikzpicture}
  \end{onlyenv}
\end{frame}

%%%%%%%%%%%%%%%%%%%%%%%%%%%%%%%%%%%%%%%%%%%%%%%%%%%%%%%%%%%%%%%%%%%%%%%%%%%%%%%%
\begin{frame}
  \frametitle{Goals of Reproducibility}
  \Huge
  \begin{enumerate}
   \item Dispel Doubts
   \item Facilitate Further Experimentation
  \end{enumerate}
  \vfill
  \footnotesize{Idea from \lhref{https://elephly.net/downies/2023-dfn-slides.pdf}{DFN slides}.}
\end{frame}

%%%%%%%%%%%%%%%%%%%%%%%%%%%%%%%%%%%%%%%%%%%%%%%%%%%%%%%%%%%%%%%%%%%%%%%%%%%%%%%%
\begin{frame}
  \frametitle{Reproducible Data Analysis}
  \begin{onlyenv}<1| handout:0>
    \begin{tikzpicture}
      \path[use as bounding box] (0.7,0) rectangle (12,8);
      \node at (5.5, 5.5) {\includegraphics[width=0.7\textwidth]{Snakemake/automation.png}};
      \node at (8, 2.5) {\begin{minipage}{0.65\textwidth}
                             \textbf{From raw data to final figures:}
                             \begin{itemize}
                                \item \textbf{document} parameters, tools, versions
                                \item \textbf{execute} without manual intervention
                              \end{itemize}
                           \end{minipage}
                           };
    \end{tikzpicture}
  \end{onlyenv}
  \begin{onlyenv}<2| handout:0>
    \begin{tikzpicture}
      \path[use as bounding box] (0.7,0) rectangle (12,8);
      \node at (5.5, 5.5) {\includegraphics[width=0.7\textwidth]{Snakemake/scalability.png}};
      \node at (8, 2.5) {\begin{minipage}{0.65\textwidth}
                             \textbf{Handle parallelization:}
                             \begin{itemize}
                                \item execute for tens of thousands of datasets
                                \item efficiently use any computing platform
                              \end{itemize}
                           \end{minipage}
                           };
    \end{tikzpicture}
  \end{onlyenv}
  \begin{onlyenv}<3| handout:1>
    \begin{tikzpicture}
      \path[use as bounding box] (0.7,0) rectangle (12,8);
      \node at (5.5, 5.5) {\includegraphics[width=0.7\textwidth]{Snakemake/portability.png}};
      \node at (8, 2.5) {\begin{minipage}{0.65\textwidth}
                             \textbf{Handle deployment:}\newline
                             be able to easily execute analyses on a different system/platform/infrastructure
                           \end{minipage}
                           };
    \end{tikzpicture}
  \end{onlyenv}
\end{frame}

%%%%%%%%%%%%%%%%%%%%%%%%%%%%%%%%%%%%%%%%%%%%%%%%%%%%%%%%%%%%%%%%%%%%%%%%%%%%%%%%
\begin{frame}
  \frametitle{Beyond Reproducibility}
  \begin{onlyenv}<1| handout:0>
    \begin{figure}
      \centering
      \includegraphics[width=0.85\textwidth]{Snakemake/reproducibility_only.png}
    \end{figure}
  \end{onlyenv}
  \begin{onlyenv}<2| handout:0>
    \begin{figure}
      \centering
      \includegraphics[width=0.85\textwidth]{Snakemake/reproducibility_empty.png}
    \end{figure}
  \end{onlyenv}
  \begin{onlyenv}<3| handout:0>
    \begin{figure}
      \centering
      \includegraphics[width=0.85\textwidth]{Snakemake/reproducibility_left.png}
    \end{figure}
  \end{onlyenv}
    \begin{onlyenv}<4| handout:1>
      \begin{figure}
        \centering
        \includegraphics[width=0.85\textwidth]{Snakemake/reproducibility_full.png}
      \end{figure}
  \end{onlyenv}
  \footnotesize{\lhref{https://doi.org/10.12688/f1000research.29032.2}{From the official \Snakemake-paper.}}
\end{frame}


%%%%%%%%%%%%%%%%%%%%%%%%%%%%%%%%%%%%%%%%%%%%%%%%%%%%%%%%%%%%%%%%%%%%%%%%%%%%%%%%
\subsection{Goals, Background \& Outline}

%%%%%%%%%%%%%%%%%%%%%%%%%%%%%%%%%%%%%%%%%%%%%%%%%%%%%%%%%%%%%%%%%%%%%%%%%%%%%%%%
\begin{frame}
  \frametitle{Questions}
  \begin{question}[The questions you most probably have when starting your Analysis:]
  	\begin{itemize}
      \item How to start quickly (with the lowest amount of overhead)?
      \item What are the necessary tools?
    \end{itemize}
  \end{question}
                                                                               
  \begin{question}[Our question to you:]
  	 How do you get this information? And: Is reproducibility and sustainability your concern?
  \end{question}
  \pause
  \begin{block}{Most frequent Sources}
   \begin{itemize}
    \item Your labmate(s)
    \item The Internet
    \item Yes, of course ... eventually, when I brag about my paper/thesis.
   \end{itemize}
  \end{block}
\end{frame}

%%%%%%%%%%%%%%%%%%%%%%%%%%%%%%%%%%%%%%%%%%%%%%%%%%%%%%%%%%%%%%%%%%%%%%%%%%%%%%%%
\begin{frame}
  \frametitle{The Workflow Approach}
  Workflow Engines answer these questions directly by providing
  \begin{itemize}
   \item entire Workflows can be selected and can be put to action.
   \item executing routines reliably.
  \end{itemize}
\end{frame}

%%%%%%%%%%%%%%%%%%%%%%%%%%%%%%%%%%%%%%%%%%%%%%%%%%%%%%%%%%%%%%%%%%%%%%%%%%%%%%%%
\begin{frame}
  \frametitle{Going HPC}
  \begin{question}
  	Why would you want to work on a cluster?
  \end{question}
  \pause
  Answers may include:
  \begin{itemize}[<+->]
   \item compute power and ressources for big data
   \item launching scalable (and otherwise portable) workflows with workflow engines
  \end{itemize}
\end{frame}
