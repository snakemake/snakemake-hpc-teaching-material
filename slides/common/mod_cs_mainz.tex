%%%%%%%%%%%%%%%%%%%%%%%%%%%%%%%%%%%%%%%%%%%%%%%%%%%%%%%%%%%%%%%%%%%%%%%%%%%%%%%%
\section{Using Code-Studio on \texttt{Mogon-NHR}}
{   
	\usebackgroundtemplate{
		\vbox to \paperheight{\vfil\hbox to \paperwidth{\hfil\includegraphics[height=\paperheight]{humor/scientist-coding.jpg}\hfil}\vfil}
		% source: https://en.m.wikipedia.org/wiki/File:Text-x-python.svg
	}
	\frame{
		\frametitle{Using Code-Studio}
		\begin{mdframed}[tikzsetting={draw=white,fill=white,fill opacity=0.8,
				line width=0pt},backgroundcolor=none,leftmargin=0,
			rightmargin=150,innertopmargin=4pt,roundcorner=10pt]
			\tableofcontents[currentsection,sections={1-4},hideothersubsections]
		\end{mdframed}
	     \vspace{18mm}
	    \hfill{\tiny \lhref{https://zenodo.org/records/11147887}{\color{white}own creation with Stable Diffusion}}
	}
   
}

%%%%%%%%%%%%%%%%%%%%%%%%%%%%%%%%%%%%%%%%%%%%%%%%%%%%%%%%%%%%%%%%%%%%%%%%%%%%%%%%
\begin{frame}
	\frametitle{What is this about?}
	\begin{question}[Questions]\begin{itemize}
			\item What is an Editor?
			\item How do I select one?
			\item How do I use it?
		\end{itemize}
	\end{question}
	\begin{docs}[Objectives]
		\begin{enumerate}
			\item Explain, why you need an editor?
			\item What is an IDE, what an editor?
			\item Starting and using \texttt{gedit} as a beginner level editor.
		\end{enumerate}
	\end{docs}
\end{frame}

%%%%%%%%%%%%%%%%%%%%%%%%%%%%%%%%%%%%%%%%%%%%%%%%%%%%%%%%%%%%%%%%%%%%%%%%%%%%%%%%
\begin{frame}
  \frametitle{Editor vs. IDEs (Integrated Development Environments)}
  \footnotesize
  \begin{columns}[t]
  	\begin{column}{.5\textwidth}
  	  {\normalsize\textbf Editors:}
  	  \begin{figure}[t]
  		 \includegraphics[height=3cm]{editors_and_IDEs/gedit.png}
  		 \caption*{\texttt{gedit} editor as started on Mogon.}
  	  \end{figure}
      \vfill
  	  \begin{itemize}
  	  	\item show syntax highlighting
  	  	\item can edit and save text files
  	  \end{itemize}  
  	\end{column}
  	\begin{column}{.5\textwidth}
  	  {\normalsize\textbf IDEs:}
  	  \begin{figure}[t]
  	  	\includegraphics[width=\textwidth]{editors_and_IDEs/MOD_VSC.png}
  		\caption*{\texttt{VSC} as started via MogonOnDemand.}
  	  \end{figure}
      \vfill
      \begin{itemize}
         \item everything an editor can do
         \item handle projects
         \item SCM support
         \item encapsulates environments
         \item \ldots
      \end{itemize}  
  	\end{column}
  \end{columns}
\end{frame}

%%%%%%%%%%%%%%%%%%%%%%%%%%%%%%%%%%%%%%%%%%%%%%%%%%%%%%%%%%%%%%%%%%%%%%%%%%%%%%%%
\begin{frame}[fragile]
  \frametitle{Which Editor or IDE should I be using?}
  \begin{hint}
     You may use any \emph{any} editor of your liking. These slides introduce an OnDemand code server as a common denominator.
  \end{hint}
  
  \begin{itemize}[<+->]
  	\item Navigate to \url{https://mod.hpc.uni-mainz.de} in your browser.
  	\item Select the CodeServer
  	\item From the drop down menu select your account "\verb,<++cluster.account++>,".
  	\item Reserve the job to run for 8 hours.
  	\item Number of Tasks needs to be 1.
  	\item CPUs per Tasks needs to be 1.
  	\item Memory should be 2 GB.
  \end{itemize}
  \pause
  \begin{hint}
  	After a while you read a button "Connect to VS Code". Follow the instruction on-screen.
  \end{hint}
\end{frame} 

