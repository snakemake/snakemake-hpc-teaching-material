%%%%%%%%%%%%%%%%%%%%%%%%%%%%%%%%%%%%%%%%%%%%%%%%%%%%%%%%%%%%%%%%%%%%%%%%%%%%%%%%
\section{Evaluating Reports}
{   
	\usebackgroundtemplate{
		\vbox to \paperheight{\vfil\hbox to \paperwidth{\hfil\includegraphics[height=.7\paperheight]{logos/report.jpg}\hfil}\vfil}
	}
	\frame{
		\frametitle{Report your Analysis Results!}
		\begin{mdframed}[tikzsetting={draw=white,fill=white,fill opacity=0.8,
				line width=0pt},backgroundcolor=none,leftmargin=0,
			rightmargin=150,innertopmargin=4pt,roundcorner=10pt]
			\tableofcontents[currentsection,sections={1-4},hideothersubsections]
		\end{mdframed}
	}
}

%%%%%%%%%%%%%%%%%%%%%%%%%%%%%%%%%%%%%%%%%%%%%%%%%%%%%%%%%%%%%%%%%%%%%%%%%%%%%%%%
\begin{frame}
  \frametitle{What is this about?}
   \begin{question}[Questions]
   	 \begin{itemize}
        \item When I want to write a publication, which software(s) version(s) have I been using?
        \item Where are my figures???
     \end{itemize}
   \end{question}
   \begin{docs}[Objectives]
   	  \begin{enumerate}
         \item Learning how to generate and interpret reports with \Snakemake{}.
      \end{enumerate}
  \end{docs}
\end{frame}

%%%%%%%%%%%%%%%%%%%%%%%%%%%%%%%%%%%%%%%%%%%%%%%%%%%%%%%%%%%%%%%%%%%%%%%%%%%%%%%%
\subsection{Reporting}

%%%%%%%%%%%%%%%%%%%%%%%%%%%%%%%%%%%%%%%%%%%%%%%%%%%%%%%%%%%%%%%%%%%%%%%%%%%%%%%%
\begin{frame}[fragile]
  \frametitle{\Snakemake{} Reports}
  Asking \Snakemake{} for a report on a completed workflow is as easy as:
  \begin{lstlisting}[language=Bash, style=Shell]
$ snakemake --report
  \end{lstlisting}
  This will generate a file called ``\altverb{report.html}'', which you can visualize with a browser.
\end{frame} 


%%%%%%%%%%%%%%%%%%%%%%%%%%%%%%%%%%%%%%%%%%%%%%%%%%%%%%%%%%%%%%%%%%%%%%%%%%%%%%%%
\begin{frame}[fragile]
  \frametitle{\Snakemake{} Reports - adding Output}
  \begin{docs}
  	 Each output file that shall be part of the report has to be marked with the \altverb{report} flag, which optionally points to a caption in \lhref{https://docutils.sourceforge.io/docs/user/rst/quickstart.html}{restructured text format}.
  \end{docs}
  An example for our workflow would be:
  \begin{lstlisting}[language=Python,style=Python]
rule plot_quals:
    input:
        "calls/all.vcf"
    output:
        @report("plots/quals.svg",@ 
               @caption="report/qual.rst")@
    script:
        "scripts/plot-quals.py"
  \end{lstlisting}
\end{frame}

%%%%%%%%%%%%%%%%%%%%%%%%%%%%%%%%%%%%%%%%%%%%%%%%%%%%%%%%%%%%%%%%%%%%%%%%%%%%%%%%
\begin{frame}[fragile]
  \frametitle{\HandsOn{Writing an Annotation}}
  Let's write the file \altverb{report/qual.rst}! It shall contain our caption.
  \pause
  Our solution might(!) look like this:
  \begin{lstlisting}
Number of variations (deviations from reference) 
per experimental record. 
  \end{lstlisting}	
\end{frame}

%%%%%%%%%%%%%%%%%%%%%%%%%%%%%%%%%%%%%%%%%%%%%%%%%%%%%%%%%%%%%%%%%%%%%%%%%%%%%%%%
\begin{frame}[fragile]
	\frametitle{\HandsOn{Adding our Figure to the Report}}
	Now add the highlighted lines:
	\begin{lstlisting}[language=Python,style=Python]
rule plot_quals:
    input:
	    "calls/all.vcf"
	output:
		@report("plots/quals.svg",@ 
		@caption="report/qual.rst")@
	script:
		"scripts/plot-quals.py"
	\end{lstlisting}
    \begin{task}{Re-Run our report generator:}
    	\altverb{snakemake --report}
    \end{task}
\end{frame}


