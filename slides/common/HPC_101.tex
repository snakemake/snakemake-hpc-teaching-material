%%%%%%%%%%%%%%%%%%%%%%%%%%%%%%%%%%%%%%%%%%%%%%%%%%%%%%%%%%%%%%%%%%%%%%%%%%%%%%%%
\section{How does Clustercomputing work?}
{   
	\usebackgroundtemplate{
		\vbox to \paperheight{\vfil\hbox to \paperwidth{\hfil\includegraphics[height=\paperheight]{misc/cluster_kit}\hfil}\vfil}
		% source: https://en.m.wikipedia.org/wiki/File:Text-x-python.svg
	}
	\frame{
		\frametitle{Running ordinary Batch Scripts}
		\begin{mdframed}[tikzsetting={draw=white,fill=white,fill opacity=0.8,
				line width=0pt},backgroundcolor=none,leftmargin=0,
			rightmargin=150,innertopmargin=4pt,roundcorner=10pt]
			\tableofcontents[currentsection,sections={1-4},hideothersubsections]
		\end{mdframed}
	}
}


%%%%%%%%%%%%%%%%%%%%%%%%%%%%%%%%%%%%%%%%%%%%%%%%%%%%%%%%%%%%%%%%%%%%%%%%%%%%%%%%
\begin{frame}
  \frametitle{What is this about?}
    \begin{question}
   	  \begin{itemize}
        \item How does ordinary job submission work on a cluster?
        \item How does it work using Snakemake? (Which parameterization is necessary?)
      \end{itemize}
   \end{question}
   \begin{docs}[Objectives]
   	 \begin{enumerate}
   	 	\item Get a feeling for the SLURM batch system.
   	 	\item We want to give you an idea of building a workflow with \emph{pure} batch system commands, first.
     \end{enumerate}
   \end{docs}
\end{frame}


%%%%%%%%%%%%%%%%%%%%%%%%%%%%%%%%%%%%%%%%%%%%%%%%%%%%%%%%%%%%%%%%%%%%%%%%%%%%%%%%
\subsection*{The \slurm Scheduler}

%%%%%%%%%%%%%%%%%%%%%%%%%%%%%%%%%%%%%%%%%%%%%%%%%%%%%%%%%%%%%%%%%%%%%%%%%%%%%%%% 
\begin{frame}
  \frametitle{What is a scheduler?}
  On HPC systems you do not \emph{just start}, you need a \emph{``scheduler''}.
  So, what's that?\newline
  A scheduler (or ``batch system'') on a HPC system should\ldots
  \begin{itemize}
  \item provide an interface to help defining workflows and/or job dependencies
  \item enable automatic submission of executions
  \item provide interfaces to monitor the executions
  \item prioritise the execution order of unrelated jobs
  \end{itemize}
  \begin{columns}
    \begin{column}{0.8\linewidth}
      This is an micro-intro to the \slurm batch system.
    \end{column}
    \begin{column}{0.2\linewidth}
      \begin{figure}
        \centering
        \includegraphics[height=1.5cm,width=1.5cm]{logos/slurm_logo.png}
      \end{figure}
    \end{column}
  \end{columns}
  \vfill
\end{frame}

%%%%%%%%%%%%%%%%%%%%%%%%%%%%%%%%%%%%%%%%%%%%%%%%%%%%%%%%%%%%%%%%%%%%%%%%%%%%%%%% 
\begin{frame}
  \frametitle{Promises, promises and even more promises}
  How does a scheduler work?
  \pause
  \begin{block}{You tell it\ldots}
    \begin{itemize}
    \item how much memory (RAM, scratch space) your job will need.\pause
    \item how much time you will spend on it.\pause
    \item how many CPUs you will need (and in which combination).\pause
    \item whether you need something special (e.g. a GPU).
    \end{itemize}
  \end{block}
  \pause \vspace{-0.2cm}
  \begin{exampleblock}{The scheduler will act:}
    \begin{itemize}
    \item It will queue up your job (and decide when it will start relative to others).\pause
    \item It will decide where your job will run physically (which hosts).\pause
    \item Eventually it will start your job (if everything was correct).
    \end{itemize}
  \end{exampleblock}
  \vfill
\end{frame}

%%%%%%%%%%%%%%%%%%%%%%%%%%%%%%%%%%%%%%%%%%%%%%%%%%%%%%%%%%%%%%%%%%%%%%%%%%%%%%% 
\input{<++course.hello_world_script++>}

%%%%%%%%%%%%%%%%%%%%%%%%%%%%%%%%%%%%%%%%%%%%%%%%%%%%%%%%%%%%%%%%%%%%%%%%%%%%%%%% 
\begin{frame}[fragile]%
	\frametitle{Please Imaging \ldots}
	Now, thing of your analysis workflow: QC, preprocessing, processing, analysis, post-processing and plotting and \ldots \newline
	\pause
	All this \emph{can} be achieved with SLURM, too all with bash:
	\begin{lstlisting}[language=Bash, style=Shell]
# First, do some pre-processing for the first job.
...
# Then, submit a job without dependencies.
jid1=$(sbatch ... job1.sh)
# NOTE: ALL 'job*sh' scripts are bash scripts,
#       with more logic than the "hello world" script.		

# Next, do some more logic as pre-processing for the 
# follow-up jobs. ...
# multiple jobs can depend on a single job
jid2=$(sbatch --dependency=afterany:$jid1 ... job2.sh)
jid3=$(sbatch --dependency=afterany:$jid1 ... job3.sh)
	\end{lstlisting}
    etc. can easily be a few thousand lines for \emph{every} workflow.
	\vfill
\end{frame}


%%%%%%%%%%%%%%%%%%%%%%%%%%%%%%%%%%%%%%%%%%%%%%%%%%%%%%%%%%%%%%%%%%%%%%%%%%%%%%%% 
\begin{frame}
  \frametitle{End of HPC Intro}
  We could co on with \emph{many} details with regards to the scheduler, the file system, etc..
  \begin{block}{HPC Courses}
   HPC teams offers courses to:
   \begin{itemize}
    \item HPC Intro
    \item Bash Intro
    \item Research Data Management
    \item many other topics
   \end{itemize}
  \end{block}
  \pause
  \begin{hint}[What's next]
      We are going to parameterize our workflow\textbf{s} for clusters and for our applications in \Snakemake!
  \end{hint}
\end{frame}





