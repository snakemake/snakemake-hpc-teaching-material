\usepackage{etoolbox}
%\setbeamertemplate{mini frames}[box]
\usepackage{babel}
\usepackage[utf8]{inputenc}
\usepackage[T1]{fontenc}
\usepackage{amsfonts,amsmath,amssymb}
\usepackage{textgreek}
\usepackage{wrapfig}

\usepackage[load-configurations=binary,binary-units=true]{siunitx}
\usepackage[normalem]{ulem} % for strikethrough with \sout

\usepackage{color,colortbl}
\usepackage{upquote}

\usepackage{pifont}

\definecolor{pblue}{RGB}{45,106,148}
\definecolor{pdarkblue}{RGB}{35,71,100}
\definecolor{plightblue}{RGB}{90,159,212}
\definecolor{pyellow}{RGB}{255,212,59}
\definecolor{pdarkyellow}{RGB}{255,188,41}
\definecolor{orange}{RGB}{255,165,0}
\definecolor{plightyellow}{RGB}{255,232,115}
\definecolor{pdarkgrey}{RGB}{100,100,100}
\definecolor{pgrey}{RGB}{153,153,153}
\definecolor{plightgrey}{RGB}{233,233,233}
\definecolor{plightgrey2}{RGB}{247,247,247}
\definecolor{pnavy}{RGB}{0,0,170}
\definecolor{BrickRed}{RGB}{150,22,11}
\definecolor{BlueViolet}{RGB}{138, 43, 226}
\definecolor{PineGreen}{RGB}{0, 51, 0}
\definecolor{light-gray}{gray}{0.95}

\definecolor{UniRot}{RGB}{193,0,42}
\definecolor{UniDunkelGrau}{RGB}{99,99,99}
\definecolor{UniHellGrau}{RGB}{172,172,172}

\definecolor{UrlColor}{rgb}{0,0.08,0.45}
\definecolor{links}{rgb}{0,0,0}

\usetheme{<++layout.theme++>} % Pittsburgh, CambridgeUS
\usecolortheme{<++layout.colortheme++>} %wolverine | crane | beaver | seahorse
\useinnertheme{rounded} 
\useoutertheme{default}
\usefonttheme{default}
%\setbeamercovered{transparent}
\setbeamertemplate{footline}[frame number]

% remove the navigation symbols
\setbeamertemplate{navigation symbols}{}

% side margins
\setbeamersize{text margin left=0.5cm, text margin right=0.5cm}

\setbeamercolor{structure}{<++layout.beamercolor_structure++>}% to modify  immediately all palettes
\setbeamercolor{title}{<++layout.beamercolor_title++>}
\setbeamercolor{title in head/foot}{<++layout.beamercolor_title_head++>}

\setbeamercolor{block title}{<++layout.beamercolor_block_title++>}
\setbeamercolor{block body}{<++layout.beamercolor_block_body++>}

% \setbeamercolor{block title}{fg=white,bg=orange}
\setbeamercolor{block title alerted}{<++layout.beamercolor_block_title++>}
\setbeamercolor{block title example}{<++layout.beamercolor_block_title++>}

% enables two line cols in tabular envs
\newcommand{\specialcell}[2][c]{%
  \begin{tabular}[#1]{@{}c@{}}#2\end{tabular}}
\usepackage{subfig}
\usepackage{tikz}
\usetikzlibrary{arrows,shapes,snakes,backgrounds,positioning,shadows,decorations,trees,decorations.pathreplacing, graphs}
\usepackage{tkz-graph}

%\usepackage{tikzpeople}
\usepackage{smartdiagram}
\usesmartdiagramlibrary{additions}

%\usepackage{mdframed}

\usepackage{adjustbox} % to adjust tikzpictures within slides


\addtobeamertemplate{footline}{}{%
\begin{tikzpicture}[remember picture,overlay]
\node[anchor=south west,yshift=2pt] at (current page.south west) {\includegraphics[height=0.8cm]{../images/logos/zdv_logo.png}};
\end{tikzpicture}}

\usepackage[tikz]{bclogo}
\newenvironment{task}[1][Task]{\bclogo[arrondi=0.1,logo=\bcoutil]{#1}}{\endbclogo}
\newenvironment{docs}[1][Documentation]{\bclogo[arrondi=0.1,logo=\bcplume]{#1}}{\endbclogo}
\newenvironment{hint}[1][Hint]{\bclogo[arrondi=0.1,logo=\bcinfo]{#1}}{\endbclogo}
\newenvironment{warning}[1][Warning]{\bclogo[arrondi=0.1,logo=\bcattention]{#1}}{\endbclogo}
% ``d/Definition'' is already defined ;-)
\newenvironment{explanation}[1][Definition]{\bclogo[arrondi=0.1,logo=\bcplume]{#1}}{\endbclogo}
\newenvironment{question}[1][Question]{\bclogo[arrondi=0.1,logo=\bcquestion]{#1}}{\endbclogo}


%%%%%%%%%%%%%%%%%
%% PLEASE NOTE %%
%%%%%%%%%%%%%%%%%
% frames containing ``Hand Out'' or ``Interlude'' should be started:

% \setcounter{preframe_handson}{\value{handson}}
% \begin{frame}[fragile]
%   \setcounter{handson}{\value{preframe_handson}}
%   \frametitle{\HandsOn{Using \texttt{find}}}

% or

% \setcounter{preframe_interlude}{\value{interlude}}
% \begin{frame}[fragile]
%   \setcounter{interlude}{\value{preframe_interlude}}
%   \frametitle{Interlude -- Parameter Extension}

% respectively.

\newcounter{handson}
\setcounter{handson}{1}
\newcounter{preframe_handson}
\setcounter{preframe_handson}{1}
\newcommand{\HandsOn}[1]{Hands On \Roman{handson} -- #1 \addtocounter{handson}{1}}
%\newcommand{\HandsOn}[1]{Hands On -- #1}

%TODO: Merge ``HandsOn'' && ``Excercise''
\newcounter{exercise}
\setcounter{exercise}{1}
% \newcommand{\Exercise}{\theexercise . Excercise \addtocounter{exercise}{1}}

% Bugfix of the Exercise command: avoid the annoying counter
\newcommand{\Exercise}{\theexercise . Excercise \addtocounter{exercise}{1}}

\newcounter{interlude}
\setcounter{interlude}{1}
\newcounter{preframe_interlude}
\setcounter{preframe_interlude}{1}

%\newcommand{\Interlude}[1]{Interlude \Roman{interlude} -- #1 \addtocounter{interlude}{1}}

% Bugfix of the Interlude command: avoid the annoying counter!
\newcommand{\Interlude}[1]{Interlude -- #1}

\usepackage{marvosym}
\usepackage{multicol}

\usepackage{hhline}

\usepackage{times}

% will decrease the font size for one frame
\newcommand\Fontvi{\fontsize{6}{7.2}\selectfont}
% 
\usepackage{dirtree,float} % for directory tree listings
\usepackage[nodisplayskipstretch]{setspace}


\usepackage{verbatim}
\usepackage{listings}

\newcommand\YAMLcolonstyle{\color{red}\mdseries}
\newcommand\YAMLkeystyle{\color{black}\bfseries}
\newcommand\YAMLvaluestyle{\color{blue}\mdseries}

\makeatletter

% here is a macro expanding to the name of the language
% (handy if you decide to change it further down the road)
\newcommand\language@yaml{yaml}

\expandafter\expandafter\expandafter\lstdefinelanguage
\expandafter{\language@yaml}
{
	keywords={true,false,null,y,n},
	keywordstyle=\color{darkgray}\bfseries,
	basicstyle=\YAMLkeystyle,                                 % assuming a key comes first
	sensitive=false,
	comment=[l]{\#},
	morecomment=[s]{/*}{*/},
	commentstyle=\color{purple}\ttfamily,
	stringstyle=\YAMLvaluestyle\ttfamily,
	moredelim=[l][\color{orange}]{\&},
	moredelim=[l][\color{magenta}]{*},
	moredelim=**[il][\YAMLcolonstyle{:}\YAMLvaluestyle]{:},   % switch to value style at :
	morestring=[b]',
	morestring=[b]",
	literate =    {---}{{\ProcessThreeDashes}}3
	{>}{{\textcolor{red}\textgreater}}1     
	{|}{{\textcolor{red}\textbar}}1 
	{\ -\ }{{\mdseries\ -\ }}3,
}

% switch to key style at EOL
\lst@AddToHook{EveryLine}{\ifx\lst@language\language@yaml\YAMLkeystyle\fi}
\makeatother

\newcommand\ProcessThreeDashes{\llap{\color{cyan}\mdseries-{-}-}}


\makeatletter
\newcommand\applyCurrentFontsize
{%
	% we first save the current fontsize, baseline-skip,
	% and listings' basicstyle
	\let\f@sizeS@ved\f@size%
	\let\f@baselineskipS@ved\f@baselineskip%
	\let\basicstyleS@ved\lst@basicstyle%
	% we now change the fontsize of listings' basicstyle
	\renewcommand\lst@basicstyle%
	{%
		\basicstyleS@ved%
		\fontsize{\f@sizeS@ved}{\f@baselineskipS@ved}%
		\selectfont%
	}%
}
\makeatother

\newcommand{\altverb}[2][{}]{\colorbox{plightgrey}{\applyCurrentFontsize \lstinline[language={#1}]{#2}}}



\lstloadlanguages{Python,bash,C++}
\lstset{showspaces=false,
basicstyle=\small,
showstringspaces=false}

\lstdefinestyle{tree}{
    literate=
    {├}{{smash{raisebox{-1ex}{rule{1pt}{baselineskip}}}raisebox{0.5ex}{rule{1ex}{1pt}}}}1 
    {─}{{raisebox{0.5ex}{rule{1.5ex}{1pt}}}}1 
    {└}{{smash{raisebox{0.5ex}{rule{1pt}{dimexprbaselineskip-1.5ex}}}raisebox{0.5ex}{rule{1ex}{1pt}}}}1 
  }

%default python listings:
\lstdefinestyle{Python}
{
  language=Python,
  basicstyle=\small,
  showstringspaces=false,
  stepnumber=5,
  numberstyle=\tiny,
  numbersep=5pt,
  showspaces=false,
  frame=single,
  framerule=0.4pt,
  rulecolor=\color{pgrey},
  backgroundcolor=\color{white},
  stringstyle=\color{BrickRed},
  keywordstyle=\color{BlueViolet}\bfseries,
  commentstyle=\color{PineGreen}\bfseries,
  identifierstyle={},
  emph={[10]self}, emphstyle={[10]\color{pblue}},
  emph={[11]yield}, emphstyle={[11]\color{pblue}},
  moredelim=**[is][\bfseries\color{red}]{@}{@},
  literate={\\@}{{\makeatletter @ \makeatother}}1
}

\lstdefinestyle{R}
{
  language=R,
  basicstyle=\small,
  showstringspaces=false,
  stepnumber=5,
  numberstyle=\tiny,
  numbersep=5pt,
  showspaces=false,
  frame=single,
  framerule=0.4pt,
  rulecolor=\color{pgrey},
  backgroundcolor=\color{white},
  stringstyle=\color{BrickRed},
  keywordstyle=\color{BlueViolet}\bfseries,
  commentstyle=\color{PineGreen}\bfseries,
  identifierstyle={},
  emph={[10]self}, emphstyle={[10]\color{pblue}},
  emph={[11]yield}, emphstyle={[11]\color{pblue}},
}

%default python listings:
\lstdefinestyle{C++}
{
  language=C++,
  basicstyle=\small,
  showstringspaces=false,
  stepnumber=5,
  numberstyle=\tiny,
  numbersep=5pt,
  showspaces=false,
  frame=single,
  framerule=0.4pt,
  rulecolor=\color{pgrey},
  backgroundcolor=\color{white},
  stringstyle=\color{BrickRed},
  keywordstyle=\color{BlueViolet}\bfseries,
  commentstyle=\color{PineGreen}\bfseries,
  identifierstyle={},
  emph={[10]self}, emphstyle={[10]\color{pblue}},
  emph={[11]yield}, emphstyle={[11]\color{pblue}},
}

\newcommand{\CC}{C\nolinebreak\hspace{-.05em}\raisebox{1ex}{\tiny\bf +}\nolinebreak\hspace{-.10em}\raisebox{1ex}{\tiny\bf +}}

%default shell listings:
\lstdefinestyle{Shell}
{
  language=Bash,
  basicstyle=\ttfamily\small,
  showstringspaces=false,
  frame=single,
  framerule=0.4pt,
  rulecolor=\color{pgrey},
  backgroundcolor=\color{plightgrey2},
  stringstyle=\color{BrickRed},
  keywordstyle=\color{BlueViolet},
  commentstyle=\color{PineGreen}\bfseries,
  identifierstyle=\color{black},
  emph={[10]\$,>>>}, emphstyle={[10]\color{pblue}},
  moredelim=**[is][\bfseries\color{red}]{@}{@},
  literate={\\@}{{\makeatletter @ \makeatother}}1
}

%default plain listings (e.g. for config files):https://www.google.com/search?client=firefox-b-e&q=conrad
\lstdefinestyle{Plain}
{ 
  stepnumber=5,
  numberstyle=\tiny,
  numbersep=5pt,
  language=Bash,
  basicstyle=\ttfamily\small,
  showstringspaces=false,
  frame=single,
  framerule=0.4pt,
  rulecolor=\color{pgrey},
  backgroundcolor=\color{plightgrey2},
  stringstyle=\color{black},
  keywordstyle=\color{black},
  commentstyle=\color{blue}\bfseries,
  identifierstyle=\color{black},
  breaklines=true,
  emph={[10]\$,>>>}, emphstyle={[10]\color{pblue}}
}

\lstdefinelanguage{XML}
{
  frame=single,
  framerule=0.4pt,
  rulecolor=\color{pgrey},
  backgroundcolor=\color{plightgrey2},
  stringstyle=\color{black},
  keywordstyle=\color{black},
  commentstyle=\color{blue}\bfseries,
  identifierstyle=\color{black},
  emph={[10]\$,>>>}, emphstyle={[10]\color{pblue}}
  morestring=[b]",
  morestring=[s]{>}{<},
  morecomment=[s]{<?}{?>},
  morekeywords={xmlns,version,type}% list your attributes here
}

\newcommand{\bibtex}{\textsc{Bib}\TeX}

%%% https://tex.stackexchange.com/questions/99316/symbol-for-external-links
\newcommand{\LinkSymbol}{%
  \tikz[x=1.2ex, y=1.2ex, baseline=-0.05ex]{% 
    \begin{scope}[x=1ex, y=1ex]
      \clip (-0.1,-0.1) 
      --++ (-0, 1.2) 
      --++ (0.6, 0) 
      --++ (0, -0.6) 
      --++ (0.6, 0) 
      --++ (0, -1);
      \path[draw, 
      line width = 0.5, 
      rounded corners=0.5] 
      (0,0) rectangle (1,1);
    \end{scope}
    \path[draw, line width = 0.5] (0.5, 0.5) 
    -- (1, 1);
    \path[draw, line width = 0.5] (0.6, 1) 
    -- (1, 1) -- (1, 0.6);
  }
}
\newcommand{\lhref}[2]{\href{#1}{#2\,\LinkSymbol}}

%%%% shortcuts for uniform appearance of common strings %%%%
\newcommand{\slurm}{\textsc{slurm}~}
\makeatletter
\newcommand{\rmnum}[1]{\romannumeral #1}
\newcommand{\Rmnum}[1]{\expandafter\@slowromancap\romannumeral #1@}
\makeatother

%%%% nicer typesetting the snakemake project
\newcommand{\Snakemake}{\mbox{
	\begingroup\normalfont
	\includegraphics[height=\texorpdfstring{\fontcharht\font`\B}]{logos/Snakemake.png}
	\textbf{Snakemake}
    \endgroup}
}


%\newcommand{\pathtoexercise}[1]{\path{/lustre/project/m2_jgu-ngstraing/workflows/#1}}
%\newcommand{\pathtoexercise}[1]{\path{ \DTLfetch{data}{thekey}{#1}{thevalue}   }}
\newcommand{\pathtoclozure}[1]{\path{workflows/tutorial/#1}}
\newcommand{\pathtosolutions}[1]{\path{/lustre/project/hpckurs/solutions/#1}}

\setcounter{tocdepth}{1}

% this allows turning of footlines for particular slides
\setbeamertemplate{footline}[frame number]
% to use it, perform:

% \begin{frame}
% normal frame
% \end{frame}
% 
% \begingroup
% \setbeamertemplate{footline}{}
% \begin{frame}
% without footline
% \end{frame}
% \endgroup

%--------------------%
% Meta-Info 
%--------------------%


\author[Snakemake Teaching Alliance]{The "Snakemake Teaching Alliance"}
\date{<++course.date++>}

\hypersetup{colorlinks,linkcolor=,urlcolor=links}

\graphicspath{{../images/}{../logos}}


% Passe captions an
\setbeamertemplate{caption}{\insertcaption}
% \setbeamerfont{caption}{size=\scriptsize}
\setlength\abovecaptionskip{-2.5pt}
\setlength\belowcaptionskip{0pt}



% For every picture that defines or uses external nodes, you'll have to
% apply the 'remember picture' style. To avoid some typing, we'll apply
% the style to all pictures.
\tikzstyle{every picture}+=[remember picture]
\tikzstyle{na} = [baseline=-.5ex]

% Add an include hook to error when a file is missing,
% to be able to recognize missing \include files on the
% CI.
% from: https://tex.stackexchange.com/questions/620515/how-to-force-latex-to-error-when-an-include-file-is-missing-misspelled
\makeatletter
\def\mkfilename#1{%
  \if\relax\detokenize\expandafter{#1}\relax\else#1/\fi}
\AddToHook{include/before}%
  {\IfFileExists{\mkfilename\CurrentFilePath\CurrentFile}{}
     {\GenericError{}{Error: File \mkfilename\CurrentFilePath\CurrentFile.tex not found!}{\@gobble}{}}}
\makeatother
