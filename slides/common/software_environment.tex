\section{Software Environment}

%%%%%%%%%%%%%%%%%%%%%%%%%%%%%%%%%%%%%%%%%%%%%%%%%%%%%%%%%%%%%%%%%%%%%%%%%%%%%%%%
\begin{frame}
	\frametitle{Outline}
	\begin{columns}[t]
		\begin{column}{.5\textwidth}
			\tableofcontents[sections={1-9},currentsection]
		\end{column}
		\begin{column}{.5\textwidth}
			\tableofcontents[sections={10-18},currentsection]
		\end{column}
	\end{columns}
\end{frame}


%%%%%%%%%%%%%%%%%%%%%%%%%%%%%%%%%%%%%%%%%%%%%%%%%%%%%%%%%%%%%%%%%%%%%%%%%%%%%%%%
\begin{frame}
	\frametitle{What is this about?}
	\begin{question}[Questions]\begin{itemize}
			\item How do I get the software for a particular workflow?
			\item What is the difference in different build systems and software environments? Why does it matter for me?
		\end{itemize}
	\end{question}
	\begin{docs}[Objectives]
	  \begin{enumerate}
			\item Introducing the "Module" system provided on HPC clusters (briefly).
			\item Learning how to install software with "Conda".
			\item Knowing how to avoid conflicts between the different software provisioning schemes.
	  \end{enumerate}
    \end{docs}
\end{frame}

%%%%%%%%%%%%%%%%%%%%%%%%%%%%%%%%%%%%%%%%%%%%%%%%%%%%%%%%%%%%%%%%%%%%%%%%%%%%%%%%
\subsection{Software on HPC Systems}

%%%%%%%%%%%%%%%%%%%%%%%%%%%%%%%%%%%%%%%%%%%%%%%%%%%%%%%%%%%%%%%%%%%%%%%%%%%%%%% 
\begin{frame}
  \frametitle{Modules}
  \vspace{-1.3em}
  \begin{block}{What is a module?}
    A module collects all environment variables and settings needed for a particular software package (e.\,g. path to executable and libraries).
  \end{block}

  \vfill
\end{frame}

%TODO: discuss: is this too dense? This works for lmod (see next slide on the spider command). Too specific?
%%%%%%%%%%%%%%%%%%%%%%%%%%%%%%%%%%%%%%%%%%%%%%%%%%%%%%%%%%%%%%%%%%%%%%%%%%%%%%%
\begin{frame}[fragile]
  {Modules -- Command Overview}
  \vspace{-1em}
  \begin{itemize}
    \setlength\itemsep{-0.1em}
  \item List of all available modules
    \begin{lstlisting}[language=Bash, style=Shell]
$ module avail             # or 'module av'
    \end{lstlisting}
  \item Loading a specific module
    \begin{lstlisting}[language=Bash, style=Shell]
$ module load <modulename> # or 'module add'
    \end{lstlisting}
  \item Showing all currently loaded modules
    \begin{lstlisting}[language=Bash, style=Shell]
$ module list
    \end{lstlisting}
  \item Unloading a specific module
    \begin{lstlisting}[language=Bash, style=Shell]
$ module unload <modulename>
    \end{lstlisting}
  \item Unload all active modules
    \begin{lstlisting}[language=Bash, style=Shell]
$ module purge
    \end{lstlisting}
  \end{itemize}
  \vfill
\end{frame}

%%%%%%%%%%%%%%%%%%%%%%%%%%%%%%%%%%%%%%%%%%%%%%%%%%%%%%%%%%%%%%%%%%%%%%%%%%%%%%%
\begin{frame}[fragile]
  \frametitle{Modules -- looking for specific modules}
  Looking up modules:
  \begin{lstlisting}[language=Bash, style=Shell]
$ module spider <search string>
  \end{lstlisting}
  \pause
  \begin{task}[Looking for area specific modules]
  	Try looking for an area specific 
    module, e.\,g. in ``\texttt{bwa}''
  \end{task}
\end{frame}

%TODO: this slides works with the naming scheme on Mogon - not necessarily on other clusters
%%%%%%%%%%%%%%%%%%%%%%%%%%%%%%%%%%%%%%%%%%%%%%%%%%%%%%%%%%%%%%%%%%%%%%%%%%%%%%%
\begin{frame}[fragile]
   {Modules with easybuild\newline Or: What is this fuzz at the end of module names?}
   You will have seen modules like:
   \begin{lstlisting}[language=Bash, style=Shell]
numlib/FFTW/3.3.10-gompi-2021b   
   \end{lstlisting}
   \pause
   \begin{block}{Easybuild Naming Scheme}
    We are building our modules with \lhref{https://easybuilders.github.io/easybuild/}{easybuild} and adopted the following naming scheme for modules:\newline
    \footnotesize \verb+<topic>/<name>/<version>-<toolchain>-<toolchain-version>+
   \end{block}
   \pause
   \begin{task}[Look inside a module to know what will be loaded and set]
   	  Do ``\altverb{module show <module>}''
   \end{task}
\end{frame}

%TODO: should we offer more or less on modules?
%%%%%%%%%%%%%%%%%%%%%%%%%%%%%%%%%%%%%%%%%%%%%%%%%%%%%%%%%%%%%%%%%%%%%%%%%%%%%%% 
\begin{frame}[fragile]
  \frametitle{That's all Folks}
   \vspace{-0.8em}
  \begin{alertblock}{Why we will not go in depth now}
You can learn more about modules in 101-HPC courses. Later, we will learn how to use \texttt{Snakemake} workflows, particularly curated ones, available on the web. We \emph{could} re-write and adapt them for Mogon, it is better to only parameterize them for Mogon and do leave the workflow itself unaltered. This is less cumbersome and as workflow systems, including \texttt{Snakemake}, rely on Conda, we will have an in-depth intro to Conda, instead.
  \end{alertblock}
  \vfill
  \begin{alertblock}{Do not mix Conda with Modules}
   Do not mix Conda with module files - particularly, avoid writing \altverb{module load} commands in your \texttt{\textasciitilde/.bashrc} file.\newline
   Whenever your modules or Conda are using conflicting compilers or environments, you might not be able to execute your software or -- \emph{worse} -- will result in funny crashes with apparently no reason.
  \end{alertblock}
\end{frame}

%%%%%%%%%%%%%%%%%%%%%%%%%%%%%%%%%%%%%%%%%%%%%%%%%%%%%%%%%%%%%%%%%%%%%%%%%%%%%%% 
\subsection{Using Conda}

%%%%%%%%%%%%%%%%%%%%%%%%%%%%%%%%%%%%%%%%%%%%%%%%%%%%%%%%%%%%%%%%%%%%%%%%%%%%%%% 
\begin{frame}<handout:0> 
  \frametitle{Your Work Environment with Conda}
  \begin{columns}
    \begin{column}{0.5\textwidth}\centering
      \includegraphics[width=0.8\textwidth]{environment/environment.png}
    \end{column}
    \begin{column}{0.5\textwidth}\centering
      \includegraphics[width=0.8\textwidth]{logos/Conda_logo.png}   
    \end{column}
  \end{columns}
\end{frame}

%%%%%%%%%%%%%%%%%%%%%%%%%%%%%%%%%%%%%%%%%%%%%%%%%%%%%%%%%%%%%%%%%%%%%%%%%%%%%%% 
\begin{frame}
  \frametitle{Conda vs. Module Files}
  \begin{columns}
    \begin{column}{0.5\textwidth}
      Background Module Files
      \begin{itemize}
       \item module files provide an environment per software
       \item the software usually is compiled on the machine (optimized)
       \item due to differences in cluster naming schemes and setups portability cannot be granted
      \end{itemize}
    \end{column}
    \begin{column}{0.5\textwidth}
      Background Conda
      \begin{itemize}
       \item Conda is a machine independent package management systems
       \item packaged software is provided pre-compiled (NOT optimized)
       \item Conda allows for grouping software stacks in environments, therefore ensuring portability
      \end{itemize}
    \end{column}
  \end{columns}
\end{frame}


%%%%%%%%%%%%%%%%%%%%%%%%%%%%%%%%%%%%%%%%%%%%%%%%%%%%%%%%%%%%%%%%%%%%%%%%%%%%%%% 
\begin{frame}[fragile]
  \frametitle{Installing Conda}
  You \emph{could} run
  \begin{lstlisting}[language=Bash, style=Shell, basicstyle=\small,breaklines=true ]
$ wget https://repo.anaconda.com/miniconda/Miniconda3-latest-Linux-x86_64.sh
  \end{lstlisting}
  to retrieve (Mini)Conda (a flavour of conda with less overhead).
  \begin{hint}
  	\footnotesize URL is from \url{https://docs.conda.io/en/latest/miniconda.html}.\newline However, we are going to use tweaked scripts provided to you.
  \end{hint}
  \pause
  \begin{hint}
  	However, instead of downloading, we will work through this together on the slides to come.
  \end{hint}
\end{frame} 

%%%%%%%%%%%%%%%%%%%%%%%%%%%%%%%%%%%%%%%%%%%%%%%%%%%%%%%%%%%%%%%%%%%%%%%%%%%%%%% 
\begin{frame}[fragile]
  \frametitle{We will favour \emph{\textmu-Mamba} over Conda!}
  Many know ``Conda'' as a package manager -- and this is correct! But:
  \begin{block}{Why we recommend using ``\textmu-Mamba''}
   Conda carries some overhead. Alternative implementations can be faster and require less files. Mamba is a ``drop-in'' for Conda. This means: \emph{every command is the same}, except we will write \altverb{mamba} where usuall \altverb{conda} would be.\newline
   Why?\newline
   Mamba is an implementation of Conda, written in \CC{}. It is able to carry out some tasks in parallel and works considerably faster. In turn, \textmu-Mamba is a staticically compiled version of Mamba and does not require a ``base'' environment (we will learn about environments, soon), which means even less overhead.
  \end{block}
\end{frame}

%%%%%%%%%%%%%%%%%%%%%%%%%%%%%%%%%%%%%%%%%%%%%%%%%%%%%%%%%%%%%%%%%%%%%%%%%%%%%%% 
\begin{frame}[fragile]
	\frametitle{Installing \sout{Conda}/\textmu-Mamba}
	Please run
	\begin{lstlisting}[language=Bash, style=Shell]
$ "${SHELL}" <(curl -L micro.mamba.pm/install.sh)
	\end{lstlisting}
\end{frame}

%%%%%%%%%%%%%%%%%%%%%%%%%%%%%%%%%%%%%%%%%%%%%%%%%%%%%%%%%%%%%%%%%%%%%%%%%%%%%%%% 
\begin{frame}
  \frametitle{Installing \sout{Conda}/\textmu-Mamba - II}
  You need to confirm (with ``Enter'')
  \begin{itemize}[<+->]
	\item the binary folder.
	\item the shell you are using (just to be on the save side)
	\item whether conda-forge shall be configured (this is a major resource channel)
	\item the prefix location (where \textmu-Mamba shall be placed)
  \end{itemize} 
  The tool will tell up about the modification of your \altverb{.bashrc} file (which is executed upon \emph{every} login, we will discuss this in a minute).
\end{frame}
	
%%%%%%%%%%%%%%%%%%%%%%%%%%%%%%%%%%%%%%%%%%%%%%%%%%%%%%%%%%%%%%%%%%%%%%%%%%%%%%% 
\begin{frame}[fragile]
  \frametitle{Installing \sout{Conda}/\textmu-Mamba - III}
  \footnotesize
  \begin{columns}[t]
    \begin{column}{0.5\textwidth}
       You now have a section like this in your ``\texttt{\textasciitilde/.bashrc}'':
       \begin{lstlisting}[language=Bash, style=Shell, basicstyle=\tiny, breaklines=true]
# >>> mamba initialize >>>
# !! Contents within this block are managed by 'mamba init' !!
export MAMBA_EXE='/home/cm/.local/bin/micromamba';
export MAMBA_ROOT_PREFIX='/home/cm/micromamba';
__mamba_setup="$("$MAMBA_EXE" shell hook --shell bash --root-prefix "$MAMBA_ROOT_PREFIX" 2> /dev/null)"
if [ $? -eq 0 ]; then
    eval "$__mamba_setup"
else
    alias micromamba="$MAMBA_EXE"  # Fallback on help from mamba activate
fi
unset __mamba_setup
# <<< mamba initialize <<<
      \end{lstlisting}
      \bcattention \emph{Every} time you log-in this will be executed. Also, here, ``\texttt{<prefix>}'' denotes \emph{your} prefix.
    \end{column}
    \begin{column}{0.5\textwidth}
       \pause
       Please edit your ``\texttt{\textasciitilde/.bashrc}'' file and put part in a function, to re-gain manual control:
       \begin{lstlisting}[language=Bash, style=Shell, basicstyle=\tiny, breaklines=true]
@function conda_initialize {@
# >>> mamba initialize >>>
# !! Contents within this block are managed by 'mamba init' !!
export MAMBA_EXE='/home/cm/.local/bin/micromamba';
export MAMBA_ROOT_PREFIX='/home/cm/micromamba';
__mamba_setup="$("$MAMBA_EXE" shell hook --shell bash --root-prefix "$MAMBA_ROOT_PREFIX" 2> /dev/null)"
if [ $? -eq 0 ]; then
    eval "$__mamba_setup"
else
    alias micromamba="$MAMBA_EXE"  # Fallback on help from mamba activate
fi
unset __mamba_setup
# <<< mamba initialize <<<
@}@
      \end{lstlisting}
      \bcattention Add the highlighted lines! and your login will be faster! Also, this allows to separate the module environment and conda-related environments safely.
    \end{column}
  \end{columns}
\end{frame}


%%%%%%%%%%%%%%%%%%%%%%%%%%%%%%%%%%%%%%%%%%%%%%%%%%%%%%%%%%%%%%%%%%%%%%%%%%%%%% 
\begin{frame}[fragile]
  \frametitle{Why this function in your \texttt{.bashrc}?}
  \begin{docs}
  	\begin{itemize}[<+->]
  		\item \emph{every} time you log in, the code in your \texttt{.bashrc} will be exectud. Depeding on your conda setup, this can be incredibly slow (another reason to use Mamba or \textmu-Mamba).
  		\item automatic inclusion of Conda/Mamba might cause interference with modules
  		\item Now, you can run \verb+conda_initialize+ in the login shell, jobs scripts, etc. upon demand and deactivate if needed.
  	\end{itemize}
  \end{docs}
\end{frame}

%%%%%%%%%%%%%%%%%%%%%%%%%%%%%%%%%%%%%%%%%%%%%%%%%%%%%%%%%%%%%%%%%%%%%%%%%%%%%%% 
\begin{frame}[fragile]
  \frametitle{Initializing Conda \& Mamba}
  To initialize Conda, simply run
  \begin{lstlisting}[language=Bash, style=Shell]
$ bash
  \end{lstlisting}
  or (better)
  \begin{lstlisting}[language=Bash, style=Shell]
$ source ~/.bashrc
  \end{lstlisting}
  Subsequently, run: 
  \begin{lstlisting}[language=Bash, style=Shell]
$ conda_initialize # if you have this function
  \end{lstlisting}
\end{frame}


%TODO: the webproxy might be different or non-existing elsewhere.
%TODO: program a cluster switch
%%%%%%%%%%%%%%%%%%%%%%%%%%%%%%%%%%%%%%%%%%%%%%%%%%%%%%%%%%%%%%%%%%%%%%%%%%%%%%% 
\begin{frame}[fragile]
  \frametitle{Reducing Search Overhead - the \texttt{.condarc}-File}
  On many HPC clusters the number of Conda channels (the repositories) is reduced by the whitelisting. Nevertheless, it helps to reduce the search time with a resource file, including  a number of definitions, \emph{before} starting:
  \begin{lstlisting}[language=Bash, style=Shell, basicstyle=\tiny]
$ cat .condarc
create_default_packages:
  - setuptools # since Python is often needed
# these channels cover most of the required software
channels:
  - conda-forge
  - bioconda
  - defaults
  - r
proxy_servers: # in case a proxy server is present
  http: http://webproxy.zdv.uni-mainz.de:8888
ssl_verify: false
auto_update_conda: false
always_yes: true # avoid confirmation(s)
env_prompt: '($(basename {default_env})) '
  \end{lstlisting}
  To obtain the same resource file, run:
  \begin{lstlisting}[language=Bash, style=Shell, basicstyle=\footnotesize]
$ cp ../setup/condarc ~/.condarc
  \end{lstlisting}
  More on \altverb{.condarc} on the \lhref{https://conda.io/projects/conda/en/latest/user-guide/configuration/use-condarc.html}{official Conda documentation site}
\end{frame}

%%%%%%%%%%%%%%%%%%%%%%%%%%%%%%%%%%%%%%%%%%%%%%%%%%%%%%%%%%%%%%%%%%%%%%%%%%%%%%% 
\begin{frame}[fragile]
  \frametitle{Searching Software with Conda v. Mamba}
  First you might want to look for software. This is done with
  \begin{lstlisting}[language=Bash, style=Shell]
$ micromamba search <softwarename>
  \end{lstlisting}
  \pause
  \begin{task}
  	Try this with a software which comes to mind.
  \end{task}
  \pause
  This will list packages with channel and version information, e.\,g.
  \begin{lstlisting}[language=Bash, style=Shell, basicstyle=\tiny]
$ micromamba search minimap
<snip>
Loading channels: done
# Name                       Version           Build  Channel             
minimap                     0.2_r124               0  bioconda            
minimap                     0.2_r124      h5bf99c6_4  bioconda
....
  \end{lstlisting}
\end{frame}


%%%%%%%%%%%%%%%%%%%%%%%%%%%%%%%%%%%%%%%%%%%%%%%%%%%%%%%%%%%%%%%%%%%%%%%%%%%%%%% 
\begin{frame}
  \frametitle{Installing Software \emph{with} Conda \& Mamba - Using Environments}
  \begin{hint}It is a good habit to have
        \begin{itemize}
          \item \emph{one} environment per workflow
          \item the environment named as the workflow
          \item this way, we have a bundle of tools, activate the environment for it
          \item \texttt{snakemake} workflows will install the tools you need for a particular workflow - only \emph{if} these tools are still missing
         \end{itemize}
  \end{hint}
  \begin{hint}[Note]
  	For now we will not use \texttt{snakemake} to install our software. This topic will be covered later.
  \end{hint}
\end{frame}

%%%%%%%%%%%%%%%%%%%%%%%%%%%%%%%%%%%%%%%%%%%%%%%%%%%%%%%%%%%%%%%%%%%%%%%%%%%%%%% 
\begin{frame}[fragile]
  \frametitle{Conda / Mamba Environments}
  With Conda/Mamba, we can activate and deactivate environments, which bundle our software. E.\,g. a software stack per workflow to ensure reproducible runs.
    \pause
  To generate a new we can run
  \begin{lstlisting}[language=Bash, style=Shell]
$ micromamba create @-n@ <environment_name>
  \end{lstlisting}
  This will create a new environment with a given name in your home directory. Alternatively, i.\,e. when dealing with file number quotas, you can point the environment to a different filesystem, e.\,g.:
  \begin{lstlisting}[language=Bash, style=Shell] 
$ micromamba create \
>  @-p@ /path/to/project/fs/<environment_name>
   \end{lstlisting}
\end{frame}

%%%%%%%%%%%%%%%%%%%%%%%%%%%%%%%%%%%%%%%%%%%%%%%%%%%%%%%%%%%%%%%%%%%%%%%%%%%%%%% 
\begin{frame}[fragile]
  \frametitle{\HandsOn{Creating your own Environment}}
  Now, create your first environment. We will
  \begin{itemize}
  	\item work in the \altverb{HOME} directory with
  	\item a \textit{named} environment
  	\item use the downloaded \altverb{environment.yaml} file to define the software stack
  	\item pre-compile all Python files with the \altverb{--pyc} flag
  \end{itemize}
  \begin{lstlisting}[language=Bash, style=Shell,breaklines=true]
$ micromamba create --pyc -f environment.yaml -n snakemake-tutorial
  \end{lstlisting}	
\end{frame}

%%%%%%%%%%%%%%%%%%%%%%%%%%%%%%%%%%%%%%%%%%%%%%%%%%%%%%%%%%%%%%%%%%%%%%%%%%%%%%% 
\begin{frame}[fragile]
  \frametitle{Listing Environments}
   Over time you will have several environments. (So far, there is one.) To list them all, simply run:
  \begin{lstlisting}[language=Bash, style=Shell]
$ micromamba env list
  \end{lstlisting}
\end{frame} 

%%%%%%%%%%%%%%%%%%%%%%%%%%%%%%%%%%%%%%%%%%%%%%%%%%%%%%%%%%%%%%%%%%%%%%%%%%%%%%% 
\begin{frame}[fragile]
  \frametitle{Creating, Activating, Deactivating Environments}
  We can create, activate and deactivate environments with
  \begin{lstlisting}[language=Bash, style=Shell]
$ micromamba create [other args] -n <environment name>
$ micromamba activate <environment name>
$ micromamba deactivate
  \end{lstlisting}
   \begin{task}
   	  Activate the environment you just created!
   \end{task}
\end{frame}

%%%%%%%%%%%%%%%%%%%%%%%%%%%%%%%%%%%%%%%%%%%%%%%%%%%%%%%%%%%%%%%%%%%%%%%%%%%%%%% 
\begin{frame}[fragile]
   \frametitle{\Interlude{Managing your Prompt}}
   \begin{question}
   	Did you notice the last line in your \texttt{.condarc} file? What does it do?
   \end{question} 
   Remember:
     \begin{lstlisting}[language=Bash, style=Shell]
$ cat .condarc
....
env_prompt: '($(basename {default_env})) '
   \end{lstlisting}
   \pause
   It causes an unnamed environment prompt to shrink, e.\,g.:
   \begin{lstlisting}[language=Bash, style=Plain]
(my_env) user@host:directory $
# instead of
(/some/long/path/my_env) user@host:directory $
   \end{lstlisting}
\end{frame}



