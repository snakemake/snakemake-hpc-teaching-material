%%%%%%%%%%%%%%%%%%%%%%%%%%%%%%%%%%%%%%%%%%%%%%%%%%%%%%%%%%%%%%%%%%%%%%%%%%%%%%%%
\section{Software Environment}
{   
	\usebackgroundtemplate{
	    \vbox to \paperheight{\vfil\hbox to \paperwidth{\hfil\includegraphics[height=.7\paperheight]{environment/environment.png}\hfil}\vfil}
    }
	\frame{
		\frametitle{Software Environment}
		\begin{mdframed}[tikzsetting={draw=white,fill=white,fill opacity=0.8,
				line width=0pt},backgroundcolor=none,leftmargin=0,
			rightmargin=150,innertopmargin=4pt,roundcorner=10pt]
			\tableofcontents[currentsection,sections={1-4},hideothersubsections]
		\end{mdframed}
	}
}

\begin{frame}
	\frametitle{What is this about?}
	\begin{question}[Questions]\begin{itemize}
			\item How do I get the software for a particular workflow?
			\item What is the difference in different build systems and software environments? Why does it matter for me?
		\end{itemize}
	\end{question}
	\begin{docs}[Objectives]
	  \begin{enumerate}
			\item Introducing the "Module" system provided on HPC clusters (briefly).
			\item Learning how to install software with "Conda".
			\item Knowing how to avoid conflicts between the different software provisioning schemes.
	  \end{enumerate}
    \end{docs}
\end{frame}

%%%%%%%%%%%%%%%%%%%%%%%%%%%%%%%%%%%%%%%%%%%%%%%%%%%%%%%%%%%%%%%%%%%%%%%%%%%%%%%%
\subsection{Software on HPC Systems}

%%%%%%%%%%%%%%%%%%%%%%%%%%%%%%%%%%%%%%%%%%%%%%%%%%%%%%%%%%%%%%%%%%%%%%%%%%%%%%% 
\begin{frame}
  \frametitle{Modules}
  \vspace{-1.3em}
  \begin{block}{What is a module?}
    A module collects all environment variables and settings needed for a particular software package (e.\,g. path to executable and libraries).
  \end{block}

  \vfill
\end{frame}

%TODO: discuss: is this too dense? This works for lmod (see next slide on the spider command). Too specific?
%%%%%%%%%%%%%%%%%%%%%%%%%%%%%%%%%%%%%%%%%%%%%%%%%%%%%%%%%%%%%%%%%%%%%%%%%%%%%%%
\begin{frame}[fragile]
  {Modules -- Command Overview}
  \vspace{-1em}
  \begin{itemize}
    \setlength\itemsep{-0.1em}
  \item List of all available modules
    \begin{lstlisting}[language=Bash, style=Shell]
$ module avail             # or 'module av'
    \end{lstlisting}
  \item Loading a specific module
    \begin{lstlisting}[language=Bash, style=Shell]
$ module load <modulename> # or 'module add'
    \end{lstlisting}
  \item Showing all currently loaded modules
    \begin{lstlisting}[language=Bash, style=Shell]
$ module list
    \end{lstlisting}
  \item Unloading a specific module
    \begin{lstlisting}[language=Bash, style=Shell]
$ module unload <modulename>
    \end{lstlisting}
  \item Unload all active modules
    \begin{lstlisting}[language=Bash, style=Shell]
$ module purge
    \end{lstlisting}
  \end{itemize}
  \vfill
\end{frame}

%%%%%%%%%%%%%%%%%%%%%%%%%%%%%%%%%%%%%%%%%%%%%%%%%%%%%%%%%%%%%%%%%%%%%%%%%%%%%%%
\begin{frame}[fragile]
  \frametitle{Modules -- looking for specific modules}
  Looking up modules:
  \begin{lstlisting}[language=Bash, style=Shell]
$ module spider <search string>
  \end{lstlisting}
  \pause
  \begin{task}[Looking for area specific modules]
  	Try looking for an area specific 
    module, e.\,g. in ``\texttt{bwa}''
  \end{task}
\end{frame}

%%%%%%%%%%%%%%%%%%%%%%%%%%%%%%%%%%%%%%%%%%%%%%%%%%%%%%%%%%%%%%%%%%%%%%%%%%%%%%% 
\begin{frame}[fragile]
  \frametitle{That's all Folks}
   \vspace{-0.8em}
  \begin{alertblock}{Why we will not go in depth now}
You can learn more about modules in 101-HPC courses. Later, we will learn how to use \Snakemake{} workflows, particularly curated ones, available on the web. We \emph{could} re-write and adapt them for a specific cluster, it is better to only parameterize them and do leave the workflow itself unaltered, portable. This is less cumbersome and as workflow systems, including \Snakemake{}, rely on Conda, we will have an in-depth intro to Conda, instead.
  \end{alertblock}
  \vfill
  \begin{alertblock}{Do not mix Conda with Modules}
   Do not mix Conda with module files - particularly, avoid writing \altverb{module load} commands in your \texttt{\textasciitilde/.bashrc} file.\newline
   Whenever your modules or Conda are using conflicting compilers or environments, you might not be able to execute your software or -- \emph{worse} -- will result in funny crashes with apparently no reason.
  \end{alertblock}
\end{frame}

%%%%%%%%%%%%%%%%%%%%%%%%%%%%%%%%%%%%%%%%%%%%%%%%%%%%%%%%%%%%%%%%%%%%%%%%%%%%%%% 
\subsection{Using Conda}

%%%%%%%%%%%%%%%%%%%%%%%%%%%%%%%%%%%%%%%%%%%%%%%%%%%%%%%%%%%%%%%%%%%%%%%%%%%%%%% 
\begin{frame}<handout:0> 
  \frametitle{Your Work Environment with Conda}
  \begin{columns}
    \begin{column}{0.5\textwidth}\centering
      \includegraphics[width=0.8\textwidth]{environment/environment.png}
    \end{column}
    \begin{column}{0.5\textwidth}\centering
      \includegraphics[width=0.8\textwidth]{logos/Conda_logo.png}   
    \end{column}
  \end{columns}
\end{frame}

%%%%%%%%%%%%%%%%%%%%%%%%%%%%%%%%%%%%%%%%%%%%%%%%%%%%%%%%%%%%%%%%%%%%%%%%%%%%%%% 
\begin{frame}
  \frametitle{Conda vs. Module Files}
  \begin{columns}[t]
    \begin{column}{0.5\textwidth}
      Background - \textbf{Module Files}
      \begin{itemize}
       \item module files provide environment variables per software
       \item usually software is compiled on the machine (optimized)
       \item software might be optimized for a network (MPI)
       \item due to differences in cluster naming schemes and setups portability cannot be granted
      \end{itemize}
    \end{column}
    \begin{column}{0.5\textwidth}
      Background - \textbf{Conda}
      \begin{itemize}
       \item Conda is a machine independent package management systems
       \item packaged software is provided pre-compiled (NOT optimized)
       \item Conda allows for grouping software stacks in environments, therefore ensuring portability
      \end{itemize}
    \end{column}
  \end{columns}
\end{frame}

%%%%%%%%%%%%%%%%%%%%%%%%%%%%%%%%%%%%%%%%%%%%%%%%%%%%%%%%%%%%%%%%%%%%%%%%%%%%%%% 
{   
	\usebackgroundtemplate{
		\vbox to \paperheight{\vfil\hbox to \paperwidth{\hfil\includegraphics[height=.7\paperheight]{environment/flavours.png}\hfil}\vfil}
	}
	\frame{
		\frametitle{Flavours}
		\begin{mdframed}[tikzsetting={draw=white,fill=white,fill opacity=0.8,
				line width=0pt},backgroundcolor=none,leftmargin=0,
			rightmargin=150,innertopmargin=4pt,roundcorner=10pt]
			Conda Implementations:
			\begin{itemize}
				\item Conda - Python
				\item Miniconda - Conda with little less overhead
				\item Mamba - drop-in replacement for "Conda", C\texttt{++}, parallel solver, fast
				\item \textmu-Mamba - static version of Mamba, little overhead, no "base" environment, slightly different commands
				\item pixi - different to Conda - not yet supported by \Snakemake.
			\end{itemize}
		\end{mdframed}
	}
}
%%%%%%%%%%%%%%%%%%%%%%%%%%%%%%%%%%%%%%%%%%%%%%%%%%%%%%%%%%%%%%%%%%%%%%%%%%%%%%% 
\begin{frame}[fragile]
  \frametitle{Installing Conda}
  You \emph{could} run
  \begin{lstlisting}[language=Bash, style=Shell, basicstyle=\tiny,breaklines=true ]
$ wget https://github.com/conda-forge/miniforge/releases/latest/download/Miniforge3-Linux-x86_64.sh
$ bash Miniforge3-Linux-x86_64.sh
  \end{lstlisting}
  to retrieve Mamba, now.\newline\pause
  \begin{hint}
  	We will work through the installation process \emph{together} on the slides to come.
  \end{hint}
  Instead, please execute the installer script
  \begin{lstlisting}[language=Bash, style=Shell]
$ bash install_conda.sh
  \end{lstlisting}
\end{frame} 

%%%%%%%%%%%%%%%%%%%%%%%%%%%%%%%%%%%%%%%%%%%%%%%%%%%%%%%%%%%%%%%%%%%%%%%%%%%%%%%% 
\begin{frame}[fragile]
  \frametitle{Conda - Installation - Part I}
  Start the installation script - if not done:
  \begin{lstlisting}[language=Bash, style=Shell]
$ bash install_conda.sh
  \end{lstlisting}
  You need to confirm (with ``Enter'')
  \begin{itemize}[<+->]
  	\item the license agreement
	\item the binary folder.
	\item the shell you are using (just to be on the save side)
  \end{itemize} 
  The tool will tell up about the modification of your \altverb{.bashrc} file (which is executed upon \emph{every} login - if not corrected).
\end{frame}
	
%%%%%%%%%%%%%%%%%%%%%%%%%%%%%%%%%%%%%%%%%%%%%%%%%%%%%%%%%%%%%%%%%%%%%%%%%%%%%%% 
\begin{frame}[fragile]
  \frametitle{Installing Conda - Part II}
  \footnotesize
  \begin{columns}[t]
    \begin{column}{0.5\textwidth}
       You now have a section like this in your ``\texttt{\textasciitilde/.bashrc}'':
       \begin{lstlisting}[language=Bash, style=Shell, basicstyle=\tiny, breaklines=true]
# !! Contents within this block are managed by 'conda init' !!
__conda_setup="$('<prefix>/bin/conda' 'shell.bash' 'hook' 2> /dev/null)"
if [ $? -eq 0 ]; then
    eval "$__conda_setup"
else
if [ -f "<prefix>/etc/profile.d/conda.sh" ]; then
    . "<prefix>/etc/profile.d/conda.sh"
else
    export PATH="<prefix>/bin:$PATH"
fi
fi
unset __conda_setup

if [ -f "<prefix>/etc/profile.d/mamba.sh" ]; then
    . "<prefix>/etc/profile.d/mamba.sh"
fi
# <<< conda initialize <<<
      \end{lstlisting}
      \bcattention \emph{Every} time you log-in this will be executed. Also, here, ``\texttt{<prefix>}'' denotes \emph{your} prefix.
    \end{column}
    \begin{column}{0.5\textwidth}
       \pause
       {\footnotesize Please edit your ``\texttt{\textasciitilde/.bashrc}'' file and put part in a function, to re-gain manual control:}
       \begin{lstlisting}[language=Bash, style=Shell, basicstyle=\tiny, breaklines=true]
@function conda_initialize {@
# !! Contents within this block are managed by 'conda init' !!
    __conda_setup="$('<prefix>/bin/conda' 'shell.bash' 'hook' 2> /dev/null)"
if [ $? -eq 0 ]; then
    eval "$__conda_setup"
else
if [ -f "<prefix>/etc/profile.d/conda.sh" ]; then
   . "<prefix>/etc/profile.d/conda.sh"
else
   export PATH="<prefix>/bin:$PATH"
fi
fi
unset __conda_setup

if [ -f "<prefix>/etc/profile.d/mamba.sh" ]; then 
    . "<prefix>/etc/profile.d/mamba.sh"
fi
# <<< conda initialize <<<
@}@
      \end{lstlisting}
    \end{column}
  \end{columns}
\end{frame}


%%%%%%%%%%%%%%%%%%%%%%%%%%%%%%%%%%%%%%%%%%%%%%%%%%%%%%%%%%%%%%%%%%%%%%%%%%%%%% 
\begin{frame}[fragile]
  \frametitle{Why this function in your \texttt{.bashrc}?}
  \begin{docs}
  	\begin{itemize}[<+->]
  		\item \emph{every} time you log in, the code in your \texttt{.bashrc} will be executed. Depending on your conda setup, this can be incredibly slow (another reason to use Mamba or \textmu-Mamba).
  		\item automatic inclusion of Conda/Mamba might cause interference with modules
  		\item Now, you can run \verb+conda_initialize+ in the login shell, jobs scripts, etc. upon demand and deactivate if needed.
  	\end{itemize}
  \end{docs}
\end{frame}

%%%%%%%%%%%%%%%%%%%%%%%%%%%%%%%%%%%%%%%%%%%%%%%%%%%%%%%%%%%%%%%%%%%%%%%%%%%%%%% 
\begin{frame}[fragile]
  \frametitle{Initializing Conda \& Mamba}
  To initialize Conda, simply run

  \begin{lstlisting}[language=Bash, style=Shell]
$ # only once - not necessary after new login
$ source ~/.bashrc 
  \end{lstlisting}
  Subsequently, run: 
  \begin{lstlisting}[language=Bash, style=Shell]
$ conda_initialize # if you have this function
  \end{lstlisting}
\end{frame}

%%%%%%%%%%%%%%%%%%%%%%%%%%%%%%%%%%%%%%%%%%%%%%%%%%%%%%%%%%%%%%%%%%%%%%%%%%%%%%% 
\input{<++course.condarcfile++>}

%%%%%%%%%%%%%%%%%%%%%%%%%%%%%%%%%%%%%%%%%%%%%%%%%%%%%%%%%%%%%%%%%%%%%%%%%%%%%%% 
\begin{frame}[fragile]
  \frametitle{Searching Software with Conda v. Mamba}
  First you might want to look for software. This is done with
  \begin{lstlisting}[language=Bash, style=Shell]
$ conda search <softwarename>
  \end{lstlisting}
  \pause
  \begin{task}
  	Try this with a software which comes to mind.
  \end{task}
  \pause
  This will list packages with channel and version information, e.\,g.
  \begin{lstlisting}[language=Bash, style=Shell, basicstyle=\tiny]
$ conda search minimap
<snip>
Loading channels: done
# Name                       Version           Build  Channel             
minimap                     0.2_r124               0  bioconda            
minimap                     0.2_r124      h5bf99c6_4  bioconda
....
  \end{lstlisting}
\end{frame}


%%%%%%%%%%%%%%%%%%%%%%%%%%%%%%%%%%%%%%%%%%%%%%%%%%%%%%%%%%%%%%%%%%%%%%%%%%%%%%% 
\begin{frame}
  \frametitle{Installing Software \emph{with} Conda \& Mamba - Using Environments}
  \begin{hint}It is a good habit to have
        \begin{itemize}
          \item \emph{one} environment per workflow
          \item the environment named as the workflow
          \item this way, we have a bundle of tools, activate the environment for it
          \item \Snakemake{} workflows will install the tools you need for a particular workflow - only \emph{if} these tools are still missing
         \end{itemize}
  \end{hint}
  \begin{hint}[Note]
  	For now we will not use \Snakemake{} to install our software. This topic will be covered later.
  \end{hint}
\end{frame}

%%%%%%%%%%%%%%%%%%%%%%%%%%%%%%%%%%%%%%%%%%%%%%%%%%%%%%%%%%%%%%%%%%%%%%%%%%%%%%% 
\begin{frame}<handout:0>
	\frametitle{Why we do not create a full Environment $\ldots$}
	$\ldots$ during the course.
	\begin{docs}[Background]
		If we begin to download and install together during a course (with potentially 20 or more participants) all our software, the installation process will take too long. Particularly during installation of R and Python packages 100.000 and more files would be created. 
	\end{docs}
\end{frame}

%%%%%%%%%%%%%%%%%%%%%%%%%%%%%%%%%%%%%%%%%%%%%%%%%%%%%%%%%%%%%%%%%%%%%%%%%%%%%%% 
\begin{frame}[fragile]
	\frametitle{A first Environment for \Snakemake!}
	\begin{warning}
		Do \emph{NOT} type along! The creation of environments is important knowledge, but takes too long in a course.
	\end{warning}
    \begin{onlyenv}<1|handout:0>
      We can create a new environment:
      \begin{lstlisting}[language=Bash, style=Shell]
$ conda @create@ ...
      \end{lstlisting}
      using the \altverb{create} keyword.
    \end{onlyenv}
    \begin{onlyenv}<2|handout:0>
    	We can create a new environment:
    	\begin{lstlisting}[language=Bash, style=Shell]
$ conda create \
> ...
    	\end{lstlisting}
    	you may write everything in \emph{one} line. The \altverb{\\} breaks a line and \altverb{>} continues it - it is only to fit everything on a slide.
    \end{onlyenv}
    \begin{onlyenv}<3|handout:0>
    	We can create a new environment:
    	\begin{lstlisting}[language=Bash, style=Shell]
$ conda create \
> @-n@ snakemake_base ...
    	\end{lstlisting}
    	\altverb{-n} denotes the name of the environment.
    \end{onlyenv}
    \begin{onlyenv}<4|handout:0>
    	We can create a new environment - by cloning an existing one:
    	\begin{lstlisting}[language=Bash, style=Shell]
$ conda create \
> -n snakemake_base \
> @--clone@ ...
    	\end{lstlisting}
    	\altverb{--clone} takes a environment name or path.
    \end{onlyenv}
    \begin{onlyenv}<5|handout:0>
    	We can create a new environment:
    	\begin{lstlisting}[language=Bash, style=Shell]
$ conda create \
> -c conda-forge -c bioconda \
> --clone @<++course.softwarepath++>@
    	\end{lstlisting}
    	.
    \end{onlyenv}
    \vfill
    \begin{docs}[What is an \emph{Environment}?]
    	With Conda/Mamba, we can activate and deactivate "environments". These bundle our software. I.\,e. a software stack per workflow to ensure reproducible runs with specific software versions.
    \end{docs}
\end{frame}

%%%%%%%%%%%%%%%%%%%%%%%%%%%%%%%%%%%%%%%%%%%%%%%%%%%%%%%%%%%%%%%%%%%%%%%%%%%%%%% 
\begin{frame}[fragile]
	\frametitle{\HandsOn{Activating Environments}}
	It is time to activate a shared environment:
	\begin{lstlisting}[language=Bash, style=Shell]
$ mamba activate <++course.softwarepath++>
	\end{lstlisting}
	Your prompt should now feature the name of the environment: \altverb{snakemake}.
\end{frame}

%%%%%%%%%%%%%%%%%%%%%%%%%%%%%%%%%%%%%%%%%%%%%%%%%%%%%%%%%%%%%%%%%%%%%%%%%%%%%%% 
\begin{frame}[fragile]
	\frametitle{\HandsOn{Listing Environments}}
	Over time you will have several environments. (So far, there is one plus the \altverb{base} env.) To list them all, simply run:
	\begin{lstlisting}[language=Bash, style=Shell]
$ mamba env list
	\end{lstlisting}
\end{frame}

%%%%%%%%%%%%%%%%%%%%%%%%%%%%%%%%%%%%%%%%%%%%%%%%%%%%%%%%%%%%%%%%%%%%%%%%%%%%%%%
\begin{frame}[fragile]
    \frametitle{\HandsOn{Checking on Installed Software}}
    \begin{lstlisting}[language=Bash, style=Shell]
$ (software_stack) mamba list | grep snakemake
    \end{lstlisting}
     And you should see
     \begin{lstlisting}[style=Plain]
snakemake
snakemake-executor-plugin-slurm
snakemake-executor-plugin-slurm-jobstep
snakemake-interface-common
snakemake-interface-executor-plugins
snakemake-interface-report-plugins
snakemake-interface-storage-plugins
snakemake-minimal
snakemake-storage-plugin-fs
     \end{lstlisting}
     \ldots along with version infos, channel source and hashsums of all the dependencies.
\end{frame}

%%%%%%%%%%%%%%%%%%%%%%%%%%%%%%%%%%%%%%%%%%%%%%%%%%%%%%%%%%%%%%%%%%%%%%%%%%%%%%% 
\begin{frame}[fragile]
   \frametitle{\Interlude{Managing your Prompt}}
   \begin{question}
   	Did you notice the last line in your \texttt{.condarc} file? What does it do?
   \end{question} 
   Remember:
     \begin{lstlisting}[language=Bash, style=Shell]
$ cat .condarc
....
env_prompt: '($(basename {default_env})) '
   \end{lstlisting}
   \pause
   It causes an unnamed environment prompt to shrink, e.\,g.:
   \begin{lstlisting}[language=Bash, style=Plain]
(my_env) user@host:directory $
# instead of
(/some/long/path/my_env) user@host:directory $
   \end{lstlisting}
   if you ever need to set up your environment with a path (\altverb{-p}) instead of a name, e.\,g. on systems with a file quota.
\end{frame}



