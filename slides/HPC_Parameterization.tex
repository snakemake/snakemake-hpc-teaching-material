%%%%%%%%%%%%%%%%%%%%%%%%%%%%%%%%%%%%%%%%%%%%%%%%%%%%%%%%%%%%%%%%%%%%%%%%%%%%%%%%
\section{Going HPC}

%%%%%%%%%%%%%%%%%%%%%%%%%%%%%%%%%%%%%%%%%%%%%%%%%%%%%%%%%%%%%%%%%%%%%%%%%%%%%%%%
\begin{frame}
  \frametitle{What is this about?}
   \question[Questions]{\begin{itemize}
                         \item How does ordinary job submission work on a cluster?
                         \item How does it work using Snakemake? (Which parameterization is necessary?)
                        \end{itemize}
                       }
   \docs[Objectives]{\begin{enumerate} 
                      \item Learn to use parameters relevant for the batch system SLURM
                     \end{enumerate}}
\end{frame}

%%%%%%%%%%%%%%%%%%%%%%%%%%%%%%%%%%%%%%%%%%%%%%%%%%%%%%%%%%%%%%%%%%%%%%%%%%%%%%%%
\section{How does Clustercomputing work?}

%%%%%%%%%%%%%%%%%%%%%%%%%%%%%%%%%%%%%%%%%%%%%%%%%%%%%%%%%%%%%%%%%%%%%%%%%%%%%%%%
\subsection*{The \slurm Scheduler}

%%%%%%%%%%%%%%%%%%%%%%%%%%%%%%%%%%%%%%%%%%%%%%%%%%%%%%%%%%%%%%%%%%%%%%%%%%%%%%%% 
\begin{frame}
  \frametitle{What is a scheduler?}
  On HPC systems you do not \emph{just start}, you need a \emph{``scheduler''}.
  So, what's that?\newline
  A scheduler (or ``batch system'') on a HPC system should\ldots
  \begin{itemize}
  \item provide an interface to help defining workflows and/or job dependencies
  \item enable automatic submission of executions
  \item provide interfaces to monitor the executions
  \item prioritise the execution order of unrelated jobs
  \end{itemize}
  \begin{columns}
    \begin{column}{0.8\linewidth}
      Since late spring 2017 we are using \slurm.
    \end{column}
    \begin{column}{0.2\linewidth}
      \begin{figure}
        \centering
        \includegraphics[height=1.5cm,width=1.5cm]{slurm_logo.png}
      \end{figure}
    \end{column}
  \end{columns}
  \vfill
\end{frame}

%%%%%%%%%%%%%%%%%%%%%%%%%%%%%%%%%%%%%%%%%%%%%%%%%%%%%%%%%%%%%%%%%%%%%%%%%%%%%%%% 
\begin{frame}
  \frametitle{Promises, promises and even more promises}
  How does a scheduler work?
  \pause
  \begin{block}{You tell it\ldots}
    \begin{itemize}
    \item how much memory (RAM, scratch space) your job will need.\pause
    \item how much time you will spend on it.\pause
    \item how many CPUs you will need (and in which combination).\pause
    \item whether you need something special (e.g. a GPU).
    \end{itemize}
  \end{block}
  \pause \vspace{-0.2cm}
  \begin{exampleblock}{The scheduler will act:}
    \begin{itemize}
    \item It will queue up your job (and decide when it will start relative to others).\pause
    \item It will decide where your job will run physically (which hosts).\pause
    \item Eventually it will start your job (if everything was correct).
    \end{itemize}
  \end{exampleblock}
  \vfill
\end{frame}

\setcounter{preframe_handson}{\value{handson}}

%%%%%%%%%%%%%%%%%%%%%%%%%%%%%%%%%%%%%%%%%%%%%%%%%%%%%%%%%%%%%%%%%%%%%%%%%%%%%%%% 
\begin{frame}[fragile]
  \setcounter{handson}{\value{preframe_handson}}
  \frametitle{\HandsOn{Your first job script}}
  \hint{From now on, we will be scripting examples (cloze-based). For this you will
        need an editor. If you do not know any other editor, use \texttt{gedit}:}
  \begin{lstlisting}[language=Bash, style=Shell, basicstyle=\scriptsize]
$ # cd into appropriate directory
$ # Start gedit with the command 
$ gedit &
  \end{lstlisting}
  \hint{The \texttt{\&} will put the editor into the background.} 
\end{frame}

%%%%%%%%%%%%%%%%%%%%%%%%%%%%%%%%%%%%%%%%%%%%%%%%%%%%%%%%%%%%%%%%%%%%%%%%%%%%%%%% 
\begin{frame}[fragile]
  \setcounter{handson}{\value{preframe_handson}}
  \frametitle{\HandsOn{Your first job script}}
  \vspace{-2em}
  \begin{minipage}[t][0.32\textheight][t]{1.0\linewidth}
  \begin{lstlisting}[language=Bash, style=Shell, basicstyle=\scriptsize]
#!/bin/bash

#SBATCH -A m2_jgu-ngstraining
#SBATCH -p smp

srun echo "Hello World from job $SLURM_JOB_ID on node $(hostname)"
\end{lstlisting}
\end{minipage}\newline
\begin{minipage}[t][0.3\textheight][t]{1.0\linewidth}
  \begin{onlyenv}<1>
    \task{
    Save the script as \texttt{hello\_world.sh} and submit it with the following statement:}
    \begin{lstlisting}[language=Bash, style=Shell, basicstyle=\footnotesize]
$ sbatch hello_world.sh
\end{lstlisting}
\end{onlyenv}
\begin{onlyenv}<2>
\begin{block}{Important Items and Aspects}
  \begin{itemize}
  \item Interpreter directive
  \item Account necessary
  \item Reservation only during course
  \item Job step with \texttt{srun}
  \item Question: Where is the output?
  \end{itemize}
\end{block}
\end{onlyenv}
\end{minipage}
\vfill
\end{frame}

%%%%%%%%%%%%%%%%%%%%%%%%%%%%%%%%%%%%%%%%%%%%%%%%%%%%%%%%%%%%%%%%%%%%%%%%%%%%%%%% 
\begin{frame}
  \frametitle{End of HPC Intro}
  We could co on with \emph{many} details with regards to the scheduler, the file system, etc..
  \begin{block}{HPC Courses}
   The HPC teams offers courses to:
   \begin{itemize}
    \item HPC Intro
    \item Bash Intro
    \item Research Data Management
    \item lots of more (hopefully)
   \end{itemize}
  \end{block}
  \pause
  \hint[What's next]{We are going to parameterize our workflow\textbf{s} for clusters and for our applications in \texttt{Snakemake}!}
\end{frame}


%%%%%%%%%%%%%%%%%%%%%%%%%%%%%%%%%%%%%%%%%%%%%%%%%%%%%%%%%%%%%%%%%%%%%%%%%%%%%%%%
\section{Parametizing your Workflow}

%%%%%%%%%%%%%%%%%%%%%%%%%%%%%%%%%%%%%%%%%%%%%%%%%%%%%%%%%%%%%%%%%%%%%%%%%%%%%%%%
\subsection{Command Line vs. Configuration File}

%%%%%%%%%%%%%%%%%%%%%%%%%%%%%%%%%%%%%%%%%%%%%%%%%%%%%%%%%%%%%%%%%%%%%%%%%%%%%%%%
\begin{frame}
  \docs{\texttt{Snakemake} has an extensive command line interface (CLI). \emph{Everything} can be configured on the command line. In addition (almost) everything can be specified in a configuration file.}
  \pause
  \begin{exampleblock}{Which parameter goes where? Some rules of thumb}
    \begin{columns}[t]
      \begin{column}{0.5\textwidth}
        The CLI:
        \begin{itemize}
         \item frequently changing parameters
         \item short parameters
         \item default parameters
        \end{itemize}
      \end{column}
      \begin{column}{0.5\textwidth}
        The Config File:
        \begin{itemize}
         \item non-volantile parameters specific to your analysis (those which merit mentioning in paper should always go into a file)
         \item long parameters
         \item otherwise workflow specific parameters
        \end{itemize}
      \end{column}
    \end{columns}
  \end{exampleblock}
\end{frame}


%%%%%%%%%%%%%%%%%%%%%%%%%%%%%%%%%%%%%%%%%%%%%%%%%%%%%%%%%%%%%%%%%%%%%%%%%%%%%%%%
\subsection{The Command Line}

%%%%%%%%%%%%%%%%%%%%%%%%%%%%%%%%%%%%%%%%%%%%%%%%%%%%%%%%%%%%%%%%%%%%%%%%%%%%%%%%
\begin{frame}[fragile]
  \frametitle{Executor Selection}
  \texttt{Snakemake} lets you select various executors. Not happy with \mogon? Take \lhref{https://snakemake.readthedocs.io/en/stable/executor_tutorial/tutorial.html}{Google Lifescience, Tibanne, Kubernetes, \ldots} \newline
  We may happily select the most prominent HPC batch system, the one running on \mogon, too:
  \begin{lstlisting}[language=Bash, style=Shell]
$ snakemake --slurm
  \end{lstlisting}
  Now, \emph{every} rule will submit its jobs as HPC compute jobs.
  \hint{We will learn how to avoid this, soon-ish.}
\end{frame}

%%%%%%%%%%%%%%%%%%%%%%%%%%%%%%%%%%%%%%%%%%%%%%%%%%%%%%%%%%%%%%%%%%%%%%%%%%%%%%%%
\begin{frame}[fragile]
  \frametitle{Default Resources for \texttt{SLURM}}
  Without specifying our SLURM-account and a (default) partition, submitting batch jobs will fail. \texttt{Snakemake} allows to define so-called default resources (using \altverb{--default-resources}). With them our minimal command line becomes:
  \begin{lstlisting}[language=Bash, style=Shell]
$ snakemake --slurm --default-resources slurm_account=m2_zdvhpc slurm_partition=smp
  \end{lstlisting}
  \hint{Please notice the missing quotation marks! All arguments belong to one parameter.}
\end{frame}



%%%%%%%%%%%%%%%%%%%%%%%%%%%%%%%%%%%%%%%%%%%%%%%%%%%%%%%%%%%%%%%%%%%%%%%%%%%%%%%%
\subsection{The Configuration File}

%%%%%%%%%%%%%%%%%%%%%%%%%%%%%%%%%%%%%%%%%%%%%%%%%%%%%%%%%%%%%%%%%%%%%%%%%%%%%%%%
\begin{frame}[fragile]
  \frametitle{The \texttt{Snakemake} \texttt{resources} Section}
  \texttt{Snakemake} rules provide an additional section:
  \begin{lstlisting}[language=Python,style=Python]
rule <name>:
   ...
   resources:
      partition='parallel',
      mem_mb=1800,
      cpus_per_task=4
  \end{lstlisting}
  \hint{Note the \textbf{,}!}
  \hint[Outlook]{This is work in development, future versions will be more tolerant, e.\,g. \texttt{threads} will translate to \texttt{cpus\_per\_task} and vice versa.}
\end{frame}

%%%%%%%%%%%%%%%%%%%%%%%%%%%%%%%%%%%%%%%%%%%%%%%%%%%%%%%%%%%%%%%%%%%%%%%%%%%%%%%%
\begin{frame}[fragile]
  \frametitle{Our Workflow}
  \task{Add the following resource to our workflow.}
  \begin{lstlisting}[language=Python,style=Python]
   resources:
      mem_mb=1800
  \end{lstlisting}
  \question{Why? Why not more?}
  \pause
  Because,
  \begin{itemize}
   \item \texttt{ntasks} and \texttt{cpus\_per\_task} default to 1, which is the case here.
   \item account and partition are the same everywhere and we can use this upon submit time (see next slide)
   \item \texttt{mem\_mb} is the same everywhere, too. But: it is usually a resource to be adapter per rule. So, we try this here, too.
  \end{itemize}
\end{frame}

%%%%%%%%%%%%%%%%%%%%%%%%%%%%%%%%%%%%%%%%%%%%%%%%%%%%%%%%%%%%%%%%%%%%%%%%%%%%%%%%
\begin{frame}[fragile]
  \frametitle{Starting our Workflow - One last Time}
  \begin{lstlisting}[language=Bash,style=Shell]
$ snakemake -j unlimited --use-envmodules --slurm \
  --default-ressources account=hpckurs partition=smp
  \end{lstlisting}
  \question{Which warning(s) do turn up? What is the remedy?}
\end{frame}

%%%%%%%%%%%%%%%%%%%%%%%%%%%%%%%%%%%%%%%%%%%%%%%%%%%%%%%%%%%%%%%%%%%%%%%%%%%%%%%%
\begin{frame}[fragile]
  \frametitle{Configuration Files}
  Workflows, once established, should not be altered. All settings go into configuration files. To indicate a configuration file, run with:
  \begin{lstlisting}[language=Bash,style=Shell]
$ snakemake --configfile=<path>
  \end{lstlisting}\pause
  The configuration file itsels is in YAML format and might look like:
  \begin{lstlisting}[language=Python,style=Python]
INPUT_DIR: "/lustre/project/..."
OUTPUT_DIR: "/lustre/project/..."
# environment modules:
VINALC: "bio/VinaLC/1.3.0-gompi-2021b"
  \end{lstlisting}\pause
    Within the Snakefile we can retrieve this information, as it is represented as Python \texttt{dicts}:
  \begin{columns}
     \begin{column}{0.5\textwidth}
       \begin{lstlisting}[language=Bash,style=Shell,basicstyle=\small]
INPUT_DIR=config["INPUT_DIR"]
OUTPU_DIR=config["OUTPUT_DIR"]
       \end{lstlisting}
     \end{column}
     \begin{column}{0.5\textwidth}
      \begin{lstlisting}[language=Bash,style=Shell,basicstyle=\small] 
rule NAME:HPC_Parameterization.tex
    ...
    envmodules:
       config["VINALC"]    
      \end{lstlisting}

     \end{column}
  \end{columns}

\end{frame}



