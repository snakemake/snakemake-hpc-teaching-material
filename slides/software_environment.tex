\section{Software Environment}

%%%%%%%%%%%%%%%%%%%%%%%%%%%%%%%%%%%%%%%%%%%%%%%%%%%%%%%%%%%%%%%%%%%%%%%%%%%%%%%%
\begin{frame}
    \frametitle{Outline}
    \begin{columns}[t]
        \begin{column}{.5\textwidth}
            \tableofcontents[sections={1-9},currentsection]
        \end{column}
        \begin{column}{.5\textwidth}
            \tableofcontents[sections={10-18},currentsection]
        \end{column}
    \end{columns}
\end{frame}

%%%%%%%%%%%%%%%%%%%%%%%%%%%%%%%%%%%%%%%%%%%%%%%%%%%%%%%%%%%%%%%%%%%%%%%%%%%%%%%% 
\begin{frame}<handout:0> 
  \frametitle{Your Work Environment}
  \centering
  \includegraphics[width=0.4\textwidth]{environment/environment.png}
\end{frame}

%%%%%%%%%%%%%%%%%%%%%%%%%%%%%%%%%%%%%%%%%%%%%%%%%%%%%%%%%%%%%%%%%%%%%%%%%%%%%%%%
\subsection{Software on HPC Systems}

%%%%%%%%%%%%%%%%%%%%%%%%%%%%%%%%%%%%%%%%%%%%%%%%%%%%%%%%%%%%%%%%%%%%%%%%%%%%%%%% 
\begin{frame}[fragile]
  \frametitle{Modules}
  \vspace{-1.3em}
  \begin{block}{What is a module?}
    A module collects all environment variables and settings needed for a particular software package (e.\,g. path to executable and libraries).
  \end{block}
  \vspace{-0.8em}
  \begin{alertblock}{Why we will not go in depth now}
You can learn more about modules in 101-HPC courses. Later, we will learn how to use \texttt{Snakemake} workflows, particularly curated ones, available on the web. We \emph{could} re-write and adapt them for Mogon, it is better to only parameterize them for Mogon and do leave the workflow itself unaltered. This is less cumbersome and as workflow systems, including \texttt{Snakemake}, rely on Conda, we will have an in-depth intro to Conda, instead.
  \end{alertblock}

  \vfill
\end{frame}




