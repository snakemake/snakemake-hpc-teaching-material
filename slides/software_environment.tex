\section{Software Environment}

%%%%%%%%%%%%%%%%%%%%%%%%%%%%%%%%%%%%%%%%%%%%%%%%%%%%%%%%%%%%%%%%%%%%%%%%%%%%%%%%
\begin{frame}
    \frametitle{Outline}
    \begin{columns}[t]
        \begin{column}{.5\textwidth}
            \tableofcontents[sections={1-9},currentsection]
        \end{column}
        \begin{column}{.5\textwidth}
            \tableofcontents[sections={10-18},currentsection]
        \end{column}
    \end{columns}
\end{frame}

%%%%%%%%%%%%%%%%%%%%%%%%%%%%%%%%%%%%%%%%%%%%%%%%%%%%%%%%%%%%%%%%%%%%%%%%%%%%%%%% 
\begin{frame}<handout:0> 
  \frametitle{Your Work Environment}
  \centering
  \includegraphics[width=0.4\textwidth]{environment/environment.png}
\end{frame}

%%%%%%%%%%%%%%%%%%%%%%%%%%%%%%%%%%%%%%%%%%%%%%%%%%%%%%%%%%%%%%%%%%%%%%%%%%%%%%%%
\subsection{Software on HPC Systems}

%%%%%%%%%%%%%%%%%%%%%%%%%%%%%%%%%%%%%%%%%%%%%%%%%%%%%%%%%%%%%%%%%%%%%%%%%%%%%%%% 
\begin{frame}
  \frametitle{Modules}
  \vspace{-1.3em}
  \begin{block}{What is a module?}
    A module collects all environment variables and settings needed for a particular software package (e.\,g. path to executable and libraries).
  \end{block}

  \vfill
\end{frame}


%%%%%%%%%%%%%%%%%%%%%%%%%%%%%%%%%%%%%%%%%%%%%%%%%%%%%%%%%%%%%%%%%%%%%%%%%%%%%%%%
\begin{frame}[fragile]
  {Modules -- Command Overview}
  \vspace{-1em}
  \begin{itemize}
    \setlength\itemsep{-0.1em}
  \item List of all available modules
    \begin{lstlisting}[language=Bash, style=Shell]
$ module avail             # or 'module av'
    \end{lstlisting}
  \item Loading a specific module
    \begin{lstlisting}[language=Bash, style=Shell]
$ module load <modulename> # or 'module add'
    \end{lstlisting}
  \item Showing all currently loaded modules
    \begin{lstlisting}[language=Bash, style=Shell]
$ module list
    \end{lstlisting}
  \item Unloading a specific module
    \begin{lstlisting}[language=Bash, style=Shell]
$ module unload <modulename>
    \end{lstlisting}
  \item Unload all active modules
    \begin{lstlisting}[language=Bash, style=Shell]
$ module purge
    \end{lstlisting}
  \end{itemize}
  \vfill
\end{frame}

%%%%%%%%%%%%%%%%%%%%%%%%%%%%%%%%%%%%%%%%%%%%%%%%%%%%%%%%%%%%%%%%%%%%%%%%%%%%%%%%
\begin{frame}[fragile]
  {Modules -- looking for specific modules}
  Looking up modules:
  \begin{lstlisting}[language=Bash, style=Shell]
$ module spider <search string>
  \end{lstlisting}
  \pause
  \task[Looking for area specific modules]{Try looking for an area specific 
    module, e.\,g. in ``\texttt{bwa}''}
\end{frame}

%%%%%%%%%%%%%%%%%%%%%%%%%%%%%%%%%%%%%%%%%%%%%%%%%%%%%%%%%%%%%%%%%%%%%%%%%%%%%%%%
\begin{frame}[fragile]
   {Modules with easybuild\newline Or: What is this fuzz at the end of module names?}
   You will have seen modules like:
   \begin{lstlisting}[language=Bash, style=Shell]
numlib/FFTW/3.3.6-gompi-2017a   
   \end{lstlisting}
   \pause
   \begin{block}{Easybuild Naming Scheme}
    We are building our modules with \lhref{https://easybuilders.github.io/easybuild/}{easybuild} and adopted the following naming scheme for modules:\newline
    \footnotesize \verb+<topic>/<name>/<version>-<toolchain>-<toolchain-version>+
   \end{block}
   \pause
   \task[Look inside a module to know what will be loaded and set]{Do ``\texttt{module show <module>}''}
\end{frame}

%%%%%%%%%%%%%%%%%%%%%%%%%%%%%%%%%%%%%%%%%%%%%%%%%%%%%%%%%%%%%%%%%%%%%%%%%%%%%%%% 
\begin{frame}
  \frametitle{Requesting software installs}
  Your favorite software solution is not installed?
  \begin{exampleblock}{Request software installation}
    Use our \lhref{https://hpc.uni-mainz.de/high-performance-computing/service-angebot/softwareinstallation/}{little form}.\newline
    We are sorry to be this formal, but otherwise we cannot easily accomodate your needs.
  \end{exampleblock}
  \pause
  \warning[Please note:]{Software not incorporated in so-called easyconfigs, cannot be installed quickly.\newline
    But we will try to do our best.}
  \vfill
\end{frame}


%%%%%%%%%%%%%%%%%%%%%%%%%%%%%%%%%%%%%%%%%%%%%%%%%%%%%%%%%%%%%%%%%%%%%%%%%%%%%%%% 
\begin{frame}[fragile]
  \frametitle{That's all Folks}
   \vspace{-0.8em}
  \begin{alertblock}{Why we will not go in depth now}
You can learn more about modules in 101-HPC courses. Later, we will learn how to use \texttt{Snakemake} workflows, particularly curated ones, available on the web. We \emph{could} re-write and adapt them for Mogon, it is better to only parameterize them for Mogon and do leave the workflow itself unaltered. This is less cumbersome and as workflow systems, including \texttt{Snakemake}, rely on Conda, we will have an in-depth intro to Conda, instead.
  \end{alertblock}
  \vfill
  \begin{alertblock}{Do not mix Conda with Modules}
   Do not mix Conda with module files - particularly, avoid writing \altverb{module load} commands in your \texttt{\textasciitilde/.bashrc} file.\newline
   Whenever your modules or Conda are using conflicting compilers or environments, you might not be able to execute your software or -- \emph{worse} -- will result in funny crashes with apparently no reason.
  \end{alertblock}

\end{frame}

\section{Using Conda}

%%%%%%%%%%%%%%%%%%%%%%%%%%%%%%%%%%%%%%%%%%%%%%%%%%%%%%%%%%%%%%%%%%%%%%%%%%%%%%%% 
\begin{frame}<handout:0> 
  \frametitle{Your Work Environment with Conda}
  \begin{columns}
    \begin{column}{0.5\textwidth}\centering
      \includegraphics[width=0.8\textwidth]{environment/environment.png}
    \end{column}
    \begin{column}{0.5\textwidth}\centering
      \includegraphics[width=0.8\textwidth]{logos/Conda_logo.png}   
    \end{column}
  \end{columns}
\end{frame}

%%%%%%%%%%%%%%%%%%%%%%%%%%%%%%%%%%%%%%%%%%%%%%%%%%%%%%%%%%%%%%%%%%%%%%%%%%%%%%%% 
\begin{frame}[fragile]
  \frametitle{Installing Conda}
  You \emph{could} run
  \begin{lstlisting}[language=Bash, style=Shell, basicstyle=\small,breaklines=true ]
$ wget https://repo.anaconda.com/miniconda/Miniconda3-latest-Linux-x86_64.sh
  \end{lstlisting}
  \hint{\footnotesize URL is from \url{https://docs.conda.io/en/latest/miniconda.html}.\newline However, we are going to use our copied script.}
  \pause
  \docs{\footnotesize Using the ``\emph{latest}'' version of the script, we ensure not to rely on a specific version. It is always possible to install specific software releases with the current version of Conda, but we will not be required to update conda immediately after installation.}
  \hint{Instead of downloading, we will work through this together on the slides to come.}
\end{frame} 

%%%%%%%%%%%%%%%%%%%%%%%%%%%%%%%%%%%%%%%%%%%%%%%%%%%%%%%%%%%%%%%%%%%%%%%%%%%%%%%% 
\begin{frame}[fragile]
  \frametitle{Installing Conda - II}
  Please and run
  \begin{lstlisting}[language=Bash, style=Shell]
$ bash ./setup/01_install_mamba.sh
  \end{lstlisting}
  We will work through the questions together, please be patient!
  %TODO: update, when pixi is there.
  \docs[Background]{We are going to use mamba instead of conda, mamba is a faster drop-in replacement for conda.}
\end{frame}

%%%%%%%%%%%%%%%%%%%%%%%%%%%%%%%%%%%%%%%%%%%%%%%%%%%%%%%%%%%%%%%%%%%%%%%%%%%%%%%% 
\begin{frame}[fragile]
  \frametitle{Installing Conda - III}
  \begin{itemize}[<+->]
   \item You need to confirm the license. Scroll with ``blank'' and confirm with ``\altverb{yes+}''.
   \item Next, you need to set the prefix. For the course you can install Mamba/Conda in your \altverb{HOME}, else, agree in your group(s) on a path in your project and use the project directory, e.\,g. 
   \begin{lstlisting}[language=Bash, style=Shell, breaklines=true ]
[/home/<username>/mambaforge] >>> /lustre/project/<myproject>/mambaforge
   \end{lstlisting}
   \item Finally, you will be asked to initialize Miniconda. Please confirm with ``\verb+yes+''.
  \end{itemize}
\end{frame}

%%%%%%%%%%%%%%%%%%%%%%%%%%%%%%%%%%%%%%%%%%%%%%%%%%%%%%%%%%%%%%%%%%%%%%%%%%%%%%%% 
\begin{frame}[fragile]
  \frametitle{Installing Conda - IV}
  \footnotesize
  \begin{columns}[t]
    \begin{column}{0.5\textwidth}
       You now have a section like this in your ``\texttt{\textasciitilde/.bashrc}'':
       \begin{lstlisting}[language=Bash, style=Shell, basicstyle=\tiny, breaklines=true]
# >>> conda initialize >>>
# !! Contents within this block are managed by 'conda init' !!
__conda_setup="$('<prefix>/bin/conda' 'shell.bash' 'hook' 2> /dev/null)"
if [ $? -eq 0 ]; then
    eval "$__conda_setup"
else
    if [ -f "<prefix>/etc/profile.d/conda.sh" ]; then
        . "<prefix>/etc/profile.d/conda.sh"
    else
        export PATH="<prefix>/bin:$PATH"
    fi
fi
unset __conda_setup
# <<< conda initialize <<<
      \end{lstlisting}
      \bcattention \emph{Every} time you log-in this will be executed. Also, here, ``\texttt{<prefix>}'' denotes \emph{your} prefix.
    \end{column}
    \begin{column}{0.5\textwidth}
       \pause
       If you like, you can put this part in a function, to re-gain manual control:
       \begin{lstlisting}[language=Bash, style=Shell, basicstyle=\tiny, breaklines=true]
@function conda_initialize {@
# >>> conda initialize >>>
# !! Contents within this block are managed by 'conda init' !!
__conda_setup="$('<prefix>/bin/conda' 'shell.bash' 'hook' 2> /dev/null)"
if [ $? -eq 0 ]; then
    eval "$__conda_setup"
else
    if [ -f "<prefix>/etc/profile.d/conda.sh" ]; then
        . "<prefix>/etc/profile.d/conda.sh"
    else
        export PATH="<prefix>/bin:$PATH"
    fi
fi
unset __conda_setup
# <<< conda initialize <<<
@}@
      \end{lstlisting}
    \end{column}
  \end{columns}

   %
\end{frame}

%%%%%%%%%%%%%%%%%%%%%%%%%%%%%%%%%%%%%%%%%%%%%%%%%%%%%%%%%%%%%%%%%%%%%%%%%%%%%%%% 
\begin{frame}[fragile]
  \frametitle{Initializing Conda}
  To initialize Conda, simply run
  \begin{lstlisting}[language=Bash, style=Shell]
$ bash
  \end{lstlisting}
  or
  \begin{lstlisting}[language=Bash, style=Shell]
$ source ~/.bashrc
$ conda_initialize # if you have this function
  \end{lstlisting}
  Then your command line should look like:
  \begin{lstlisting}[language=Bash, style=Shell]
(base) ... $
  \end{lstlisting}
  , where \altverb{(base)} is the ``base'' environment of Conda. This is your starting environment.
\end{frame}

%%%%%%%%%%%%%%%%%%%%%%%%%%%%%%%%%%%%%%%%%%%%%%%%%%%%%%%%%%%%%%%%%%%%%%%%%%%%%%%% 
\begin{frame}[fragile]
  \frametitle{Searching Software with Conda}
  First you might have to look for software. This is done with
  \begin{lstlisting}[language=Bash, style=Shell]
$ mamba search <softwarename>
  \end{lstlisting}
  \pause
  \task{Try this with a software which comes to mind.}
  \pause
  This will list packages with channel and version information, e.\,g.
  \begin{lstlisting}[language=Bash, style=Shell, basicstyle=\tiny]
$ mamba search minimap
Loading channels: done
# Name                       Version           Build  Channel             
minimap                     0.2_r124               0  bioconda            
minimap                     0.2_r124      h5bf99c6_4  bioconda
....
  \end{lstlisting}
\end{frame}

%%%%%%%%%%%%%%%%%%%%%%%%%%%%%%%%%%%%%%%%%%%%%%%%%%%%%%%%%%%%%%%%%%%%%%%%%%%%%%%% 
\begin{frame}[fragile]
  \frametitle{Reducing Search Overhead - the \texttt{.condarc}-File}
  On \mogon{} the number of Conda channels is reduced by the whitelisting. Nevertheless, it helps to have a resource file, with a number of definitions, \emph{before} starting:
  \begin{lstlisting}[language=Bash, style=Shell, basicstyle=\tiny]
$ cat .condarc
create_default_packages:
  - setuptools
channels:
  - conda-forge
  - bioconda
  - defaults
  - r
proxy_servers:
  http: http://webproxy.zdv.uni-mainz.de:8888
ssl_verify: false
auto_update_conda: false
always_yes: true # avoid confirmation(s)
  \end{lstlisting}
  To obtain the same resource file, run:
  \begin{lstlisting}[language=Bash, style=Shell, basicstyle=\footnotesize]
$ cp ~/conda/condarc ~/.condarc
  \end{lstlisting}
\end{frame}

%%%%%%%%%%%%%%%%%%%%%%%%%%%%%%%%%%%%%%%%%%%%%%%%%%%%%%%%%%%%%%%%%%%%%%%%%%%%%%%% 
\begin{frame}[fragile]
  \frametitle{Installing Software \emph{with} Conda - Using Environments}
  \hint{It is a good habit to have
        \begin{itemize}
          \item \emph{an} environment per workflow
          \item the environment named as the workflow
          \item this way, we have a bundle of tools, activate the environment for it
          \item \texttt{snakemake} workflows will install the tools you need for a particular workflow - only \emph{if} these tools are still missing
         \end{itemize}
        }
  We run:
  \begin{lstlisting}[language=Bash, style=Shell]
$ bash setup/02_tutorial_setup.sh
  \end{lstlisting}
  \pause

  To list the available environments we can run
  \begin{lstlisting}[language=Bash, style=Shell]
$ mamba info --envs
  \end{lstlisting}
\end{frame}

%%%%%%%%%%%%%%%%%%%%%%%%%%%%%%%%%%%%%%%%%%%%%%%%%%%%%%%%%%%%%%%%%%%%%%%%%%%%%%%% 
\begin{frame}[fragile]
  \frametitle{Getting the example Data}
  Again we are using a script to download and extract our example data. Please 
  \begin{lstlisting}[language=Bash, style=Shell]
$ mdkir snakemake-tutorial
$ cd snakemake-tutorial
$ bash ../setup/get_data.sh
  \end{lstlisting}

\end{frame}


%%%%%%%%%%%%%%%%%%%%%%%%%%%%%%%%%%%%%%%%%%%%%%%%%%%%%%%%%%%%%%%%%%%%%%%%%%%%%%%% 
\begin{frame}[fragile]
  \frametitle{Installing Software \emph{with} Conda - Using Environments II}
  Using our new environment we can activate it:
  \begin{lstlisting}[language=Bash, style=Shell]
$ conda activate snakemake-tutorial
  \end{lstlisting}
  and install ``\texttt{snakemake}'' by running
  \begin{lstlisting}[language=Bash, style=Shell]
$ mamba install snakemake
  \end{lstlisting}
  Now, we are all set. ``\texttt{snakemake}'' is able to install the requested software, for a well-maintained workflow.
\end{frame}
