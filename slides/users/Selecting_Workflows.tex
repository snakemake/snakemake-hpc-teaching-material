%%%%%%%%%%%%%%%%%%%%%%%%%%%%%%%%%%%%%%%%%%%%%%%%%%%%%%%%%%%%%%%%%%%%%%%%%%%%%%%%
\section{Selecting Curated Workflows}
{   
	\usebackgroundtemplate{
		\vbox to \paperheight{\vfil\hbox to \paperwidth{\hfil\includegraphics[height=.7\paperheight]{example_dags/rulegraph_complex.png}\hfil}\vfil}
	}
	\frame{
		\frametitle{Selecting Workflows}
		\begin{mdframed}[tikzsetting={draw=white,fill=white,fill opacity=0.8,
				line width=0pt},backgroundcolor=none,leftmargin=0,
			rightmargin=150,innertopmargin=4pt,roundcorner=10pt]
			\tableofcontents[currentsection,sections={1-4},hideothersubsections]
		\end{mdframed}
	}
}


%%%%%%%%%%%%%%%%%%%%%%%%%%%%%%%%%%%%%%%%%%%%%%%%%%%%%%%%%%%%%%%%%%%%%%%%%%%%%%%%
\begin{frame}
  \frametitle{What is this about?}
   \begin{question}[Questions]
   	 \begin{itemize}
       \item How do I get a workflow for a given scientific problem?
       \item How do I run such an arbitrary workflow?
     \end{itemize}
   \end{question} 
   \begin{docs}[Objectives]
   	  \begin{enumerate}
                      \item Introducing the workflow catalogue?
                      \item Learning the difference between ``curation'' (what some people think) and ``curation'' (what really works)?
      \end{enumerate}
    \end{docs}
\end{frame}  

\subsection{The Snakemake Workflow Catalogue}

%%%%%%%%%%%%%%%%%%%%%%%%%%%%%%%%%%%%%%%%%%%%%%%%%%%%%%%%%%%%%%%%%%%%%%%%%%%%%%%
\begin{frame}
 \frametitle{Selecting and Downloading from the Workflow Catalogue}
 You can find the \Snakemake{} worfkflow catalogue, \lhref{https://snakemake.github.io/snakemake-workflow-catalog/?rules=true}{here}. It makes a difference between workflows which meet best-practice criteria - and those which do not.\newline
 \begin{columns}
   \begin{column}{0.5\textwidth}
     You can download and run any workflow. \Snakemake's portability features ensure it will work everywhere.\pause
     \begin{warning}
     	Except, you most likely cannot, because of a missing cluster configuration and some missing features.
     \end{warning}
   \end{column}
   \begin{column}{0.5\textwidth}
     \includegraphics[width=\textwidth]{Snakemake/Snakemake_Workflow_Catalog.png}
   \end{column}
 \end{columns}
\end{frame}

%%%%%%%%%%%%%%%%%%%%%%%%%%%%%%%%%%%%%%%%%%%%%%%%%%%%%%%%%%%%%%%%%%%%%%%%%%%%%%%
\begin{frame}
  \frametitle{Searching \emph{your} Workflow}
   \begin{columns}
  	\begin{column}{0.5\textwidth}
  		\includegraphics[width=\textwidth]{Snakemake/Searching_Workflows_in_Catalog.png}
  	\end{column}
  	\begin{column}{0.5\textwidth}
  		You can look for 
  		\begin{itemize}
  		  \item topical keywords
  		  \item software
  		\end{itemize}
  	\end{column}
  \end{columns}
\end{frame}

%%%%%%%%%%%%%%%%%%%%%%%%%%%%%%%%%%%%%%%%%%%%%%%%%%%%%%%%%%%%%%%%%%%%%%%%%%%%%%%
\begin{frame}
	\frametitle{Workflows Compliance}
	\begin{question}[Noted this?]
	  \includegraphics[width=0.8\textwidth]{Snakemake/Snakemake_Workflow_Catalog_Categories.png}
	\end{question}
\end{frame}

%%%%%%%%%%%%%%%%%%%%%%%%%%%%%%%%%%%%%%%%%%%%%%%%%%%%%%%%%%%%%%%%%%%%%%%%%%%%%%%%
\begin{frame}[fragile]
  \frametitle{Deployment}
  Select (=click on) any desired workflow. There are three alternatives:
  \begin{enumerate}[<+->]
   \item a workflow offers a release - in which case you can download and unpack it
   \item all workflows offers a ``\altverb{git clone}'' hint
   \item or you use the \altverb{snakedeploy} to get everything you need.
  \end{enumerate}
\end{frame}

%%%%%%%%%%%%%%%%%%%%%%%%%%%%%%%%%%%%%%%%%%%%%%%%%%%%%%%%%%%%%%%%%%%%%%%%%%%%%%%%
\begin{frame}[fragile]
	\frametitle{Deployment II - Creating an Environment Fork}
	Usually, we can create an environment like:
	\begin{lstlisting}[language=Bash, style=Shell]
$ mamba create -c conda-forge -c bioconda -n snakemake \
> snakemake snakemake-executor-plugin-slurm \
> snakemake-storage-plugin-fs
    \end{lstlisting}
    This should install a \Snakemake{} environment with all necessary tools!
       
\end{frame}

%%%%%%%%%%%%%%%%%%%%%%%%%%%%%%%%%%%%%%%%%%%%%%%%%%%%%%%%%%%%%%%%%%%%%%%%%%%%%%%%
\begin{frame}[fragile]
	\frametitle{Deployment III - First Step!}
	Follow step 2 of a selected workflow usage instructions:
	\begin{columns}
		\begin{column}{0.4\textwidth}
			\centering
			\includegraphics[width=0.95\textwidth]{Snakemake/deploy_workflow}
		\end{column}
	    \begin{column}{0.6\textwidth}
	    	A usual command is:
	    	\begin{lstlisting}[language=Bash, style=Shell]
$ snakedeploy deploy-workflow \
> <URL>
	    	\end{lstlisting}
    	    This will create the directories \altverb{workflow} and \altverb{config} in your current directory.
    	    \begin{hint}
    	    	Alternatively, you may navigate to the repository of your desired workflow and download the entire workflow.
    	    \end{hint}
	    \end{column}
	\end{columns}
\end{frame}

<+++ if course.pathtosetup is defined +++>
%%%%%%%%%%%%%%%%%%%%%%%%%%%%%%%%%%%%%%%%%%%%%%%%%%%%%%%%%%%%%%%%%%%%%%%%%%%%%%%%
\begin{frame}[fragile]
	\framtitle{Copy a Workflow}
	In our case, we shall copy a workflow and will not use \altverb{snakedeploy}.
	\begin{task}
		Please copy the workflow from\newline \altverb{<++course.pathtosetup++>} into your \altverb{HOME}.
	\end{task}
    \begin{hint}
    	Remember:
    	\begin{itemize}
    		\item change into your \altverb{HOME} with \altverb{cd -}
    		\item copy directories recursively with \altverb{cp -r <source> <destination>}
    	\end{itemize}
    \end{hint}
\end{frame}
<+++ else +++>
%%%%%%%%%%%%%%%%%%%%%%%%%%%%%%%%%%%%%%%%%%%%%%%%%%%%%%%%%%%%%%%%%%%%%%%%%%%%%%%%
\begin{frame}[fragile]
	\frametitle{Deployment IV - Let's do it!}
	\begin{task}
		We shall now deploy our sample workflow:
		\begin{itemize}
			\item Please create a directory \altverb{mkdir -p ~/example_workflow}
			\item Change to this directory.
			\item Deploy our sample workflow with
			\begin{lstlisting}[language=Bash, style=Shell]
$ snakedeploy deploy-workflow \
> <++course.deploy_url++>
			\end{lstlisting}
		\end{itemize}
	\end{task}
\end{frame}	
<+++ endif +++>

%%%%%%%%%%%%%%%%%%%%%%%%%%%%%%%%%%%%%%%%%%%%%%%%%%%%%%%%%%%%%%%%%%%%%%%%%%%%%%%%
\begin{frame}[fragile]
  \frametitle{Running Workflows on Cluster (or other environment)}
  Most likely a specific workflow never has been testing on \emph{your} computer before. It is almost ensured it will run on arbitrary servers, but clusters are a different story. \newline
  So
  \begin{itemize}[<+->]
   \item try to parameterize your workflow as we will learn
   \item if it gives issues and you know how to correct it, ``fork'' the worklow and create a pull request
   \item if you cannot fix it, create a bug report
  \end{itemize}
\end{frame}

%%%%%%%%%%%%%%%%%%%%%%%%%%%%%%%%%%%%%%%%%%%%%%%%%%%%%%%%%%%%%%%%%%%%%%%%%%%%%%%%
\begin{frame}
  \frametitle{\Interlude{Learn git!}}
  If you do not know what ``fork'' and ``pull request'' means, learn git!
  \begin{itemize}[<+->]
   \item there are courses
   \item and lots of online material
   \item and books
  \end{itemize}
  \pause
  \begin{warning}
  	Knowing git is essential in data analysis!
  \end{warning}
\end{frame}

%%%%%%%%%%%%%%%%%%%%%%%%%%%%%%%%%%%%%%%%%%%%%%%%%%%%%%%%%%%%%%%%%%%%%%%%%%%%%%%%
\begin{frame}[fragile]
  \frametitle{\HandsOn{Configuring the Workflow}}
  It is now time to configure and parameterize the workflow.
  \pause
  \begin{hint}
  	As the configuration is workflow dependent you need to follow instructions, now.
  \end{hint}
  \pause
  Eventually start the workflow using:
  \begin{lstlisting}[language=Bash, style=Shell]
$ snakemake --executor slurm \
> -j unlimited \
> --configfile <path to file> \
> --workflow-profile <path to directory> \
> --directory <path to your course output>
  \end{lstlisting}
\end{frame}

