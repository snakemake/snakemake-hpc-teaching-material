%%%%%%%%%%%%%%%%%%%%%%%%%%%%%%%%%%%%%%%%%%%%%%%%%%%%%%%%%%%%%%%%%%%%%%%%%%%%%%%%
\section{Going HPC}

%%%%%%%%%%%%%%%%%%%%%%%%%%%%%%%%%%%%%%%%%%%%%%%%%%%%%%%%%%%%%%%%%%%%%%%%%%%%%%%%
\begin{frame}
  \frametitle{What is this about?}
   \question[Questions]{\begin{itemize}
                         \item How does ordinary job submission work on a cluster?
                         \item How does it work using Snakemake? (Which parameterization is necessary?)
                        \end{itemize}
                       }
   \docs[Objectives]{\begin{enumerate} 
                      \item Learn to use parameters relevant for the batch system SLURM
                     \end{enumerate}}
\end{frame}

%%%%%%%%%%%%%%%%%%%%%%%%%%%%%%%%%%%%%%%%%%%%%%%%%%%%%%%%%%%%%%%%%%%%%%%%%%%%%%%%
\section{How does Clustercomputing work?}

%%%%%%%%%%%%%%%%%%%%%%%%%%%%%%%%%%%%%%%%%%%%%%%%%%%%%%%%%%%%%%%%%%%%%%%%%%%%%%%%
\subsection*{The \slurm Scheduler}

%%%%%%%%%%%%%%%%%%%%%%%%%%%%%%%%%%%%%%%%%%%%%%%%%%%%%%%%%%%%%%%%%%%%%%%%%%%%%%%% 
\begin{frame}
  \frametitle{What is a scheduler?}
  On HPC systems you do not \emph{just start}, you need a \emph{``scheduler''}.
  So, what's that?\newline
  A scheduler (or ``batch system'') on a HPC system should\ldots
  \begin{itemize}
  \item provide an interface to help defining workflows and/or job dependencies
  \item enable automatic submission of executions
  \item provide interfaces to monitor the executions
  \item prioritise the execution order of unrelated jobs
  \end{itemize}
  \begin{columns}
    \begin{column}{0.8\linewidth}
      Since late spring 2017 we are using \slurm.
    \end{column}
    \begin{column}{0.2\linewidth}
      \begin{figure}
        \centering
        \includegraphics[height=1.5cm,width=1.5cm]{slurm_logo.png}
      \end{figure}
    \end{column}
  \end{columns}
  \vfill
\end{frame}

%%%%%%%%%%%%%%%%%%%%%%%%%%%%%%%%%%%%%%%%%%%%%%%%%%%%%%%%%%%%%%%%%%%%%%%%%%%%%%%% 
\begin{frame}
  \frametitle{Promises, promises and even more promises}
  How does a scheduler work?
  \pause
  \begin{block}{You tell it\ldots}
    \begin{itemize}
    \item how much memory (RAM, scratch space) your job will need.\pause
    \item how much time you will spend on it.\pause
    \item how many CPUs you will need (and in which combination).\pause
    \item whether you need something special (e.g. a GPU).
    \end{itemize}
  \end{block}
  \pause \vspace{-0.2cm}
  \begin{exampleblock}{The scheduler will act:}
    \begin{itemize}
    \item It will queue up your job (and decide when it will start relative to others).\pause
    \item It will decide where your job will run physically (which hosts).\pause
    \item Eventually it will start your job (if everything was correct).
    \end{itemize}
  \end{exampleblock}
  \vfill
\end{frame}

\setcounter{preframe_handson}{\value{handson}}

%%%%%%%%%%%%%%%%%%%%%%%%%%%%%%%%%%%%%%%%%%%%%%%%%%%%%%%%%%%%%%%%%%%%%%%%%%%%%%%% 
\begin{frame}[fragile]
  \setcounter{handson}{\value{preframe_handson}}
  \frametitle{\HandsOn{Your first job script}}
  \hint{From now on, we will be scripting examples (cloze-based). For this you will
        need an editor. If you do not know any other editor, use \texttt{gedit}:}
  \begin{lstlisting}[language=Bash, style=Shell, basicstyle=\scriptsize]
$ # cd into appropriate directory
$ # Start gedit with the command 
$ gedit &
  \end{lstlisting}
  \hint{The \texttt{\&} will put the editor into the background.} 
\end{frame}

%%%%%%%%%%%%%%%%%%%%%%%%%%%%%%%%%%%%%%%%%%%%%%%%%%%%%%%%%%%%%%%%%%%%%%%%%%%%%%%% 
\begin{frame}[fragile]
  \setcounter{handson}{\value{preframe_handson}}
  \frametitle{\HandsOn{Your first job script}}
  \vspace{-2em}
  \begin{minipage}[t][0.32\textheight][t]{1.0\linewidth}
  \begin{lstlisting}[language=Bash, style=Shell, basicstyle=\scriptsize]
#!/bin/bash

#SBATCH -A m2_jgu-ngstraining
#SBATCH -p smp

srun echo "Hello World from job $SLURM_JOB_ID on node $(hostname)"
\end{lstlisting}
\end{minipage}\newline
\begin{minipage}[t][0.3\textheight][t]{1.0\linewidth}
  \begin{onlyenv}<1>
    \task{
    Save the script as \texttt{hello\_world.sh} and submit it with the following statement:}
    \begin{lstlisting}[language=Bash, style=Shell, basicstyle=\footnotesize]
$ sbatch hello_world.sh
\end{lstlisting}
\end{onlyenv}
\begin{onlyenv}<2>
\begin{block}{Important Items and Aspects}
  \begin{itemize}
  \item Interpreter directive
  \item Account necessary
  \item Reservation only during course
  \item Job step with \texttt{srun}
  \item Question: Where is the output?
  \end{itemize}
\end{block}
\end{onlyenv}
\end{minipage}
\vfill
\end{frame}

%%%%%%%%%%%%%%%%%%%%%%%%%%%%%%%%%%%%%%%%%%%%%%%%%%%%%%%%%%%%%%%%%%%%%%%%%%%%%%%% 
\begin{frame}
  \frametitle{End of HPC Intro}
  We could co on with \emph{many} details with regards to the scheduler, the file system, etc..
  \begin{block}{HPC Courses}
   The HPC teams offers courses to:
   \begin{itemize}
    \item HPC Intro
    \item Bash Intro
    \item Research Data Management
    \item lots of more (hopefully)
   \end{itemize}
  \end{block}
  \pause
  \hint[What's next]{We are going to parameterize our workflow\textbf{s} for clusters and for our applications in \texttt{Snakemake}!}
\end{frame}





