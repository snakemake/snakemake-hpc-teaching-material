%%%%%%%%%%%%%%%%%%%%%%%%%%%%%%%%%%%%%%%%%%%%%%%%%%%%%%%%%%%%%%%%%%%%%%%%%%%%%%%%
\section{Software Environment}
{   
	\usebackgroundtemplate{
	    \vbox to \paperheight{\vfil\hbox to \paperwidth{\hfil\includegraphics[height=.7\paperheight]{environment/environment.png}\hfil}\vfil}
    }
	\frame{
		\frametitle{Software Environment}
		\begin{mdframed}[tikzsetting={draw=white,fill=white,fill opacity=0.8,
				line width=0pt},backgroundcolor=none,leftmargin=0,
			rightmargin=150,innertopmargin=4pt,roundcorner=10pt]
			\tableofcontents[currentsection,sections={1-4},hideothersubsections]
		\end{mdframed}
	}
}

\begin{frame}
	\frametitle{What is this about?}
	\begin{question}[Questions]\begin{itemize}
			\item How does your software support integrate with\newline \Snakemake?
		\end{itemize}
	\end{question}
	\begin{docs}[Objectives]
	  \begin{enumerate}
			\item knowing software selections supported by \Snakemake
			\item avoiding software selection conflicts

	  \end{enumerate}
    \end{docs}
\end{frame}

%%%%%%%%%%%%%%%%%%%%%%%%%%%%%%%%%%%%%%%%%%%%%%%%%%%%%%%%%%%%%%%%%%%%%%%%%%%%%%%%
\begin{frame}[fragile]
	\frametitle{Snakemake's Software Provisioning - Overview}
	\Snakemake{} basically offers 3 software deployment methods:
	\begin{itemize}[<+->]
		\item Container
		\item Environment Modules
		\item Conda (and its derivatives)
	\end{itemize}
    \pause
    While Conda is the preferred and \emph{portable} way with Snakemake, other software methods can be selected with
    \begin{lstlisting}[language=Bash, style=Shell]
$ snakemake --sdm=[conda,envmodules,apptainer]
    \end{lstlisting}
    \verb+--sdm+ is short for \verb+--software-deployment-method+.
    \pause
    \begin{hint}
    	Using environment modules or container based deployment \emph{is possible}, but tedious if used for more than a \emph{few} $3^{\mathsf{rd}}$ party programs.
    \end{hint}
\end{frame}
	
%%%%%%%%%%%%%%%%%%%%%%%%%%%%%%%%%%%%%%%%%%%%%%%%%%%%%%%%%%%%%%%%%%%%%%%%%%%%%%% 
\begin{frame}[fragile]
  \frametitle{Separating Mamba or Conda from Env Modules}
  \footnotesize
  \begin{columns}[t]
    \begin{column}{0.5\textwidth}
       Installing Mamba or Conda gives this section in a user's ``\texttt{\textasciitilde/.bashrc}'':
       \begin{lstlisting}[language=Bash, style=Shell, basicstyle=\tiny, breaklines=true]
# !! Contents within this block are managed by 'conda init' !!
__conda_setup="$('<prefix>/bin/conda' 'shell.bash' 'hook' 2> /dev/null)"
if [ $? -eq 0 ]; then
    eval "$__conda_setup"
else
if [ -f "<prefix>/etc/profile.d/conda.sh" ]; then
    . "<prefix>/etc/profile.d/conda.sh"
else
    export PATH="<prefix>/bin:$PATH"
fi
fi
unset __conda_setup

if [ -f "<prefix>/etc/profile.d/mamba.sh" ]; then
    . "<prefix>/etc/profile.d/mamba.sh"
fi
# <<< conda initialize <<<
      \end{lstlisting}
      \bcattention \emph{Every} time they log-in this will be executed and will potentially interfere with "ordinary" HPC works.
    \end{column}
    \begin{column}{0.5\textwidth}
       \pause
       {\footnotesize We recommend to edit the ``\texttt{\textasciitilde/.bashrc}'' file and put a part in a function, to re-gain manual control:}
       \begin{lstlisting}[language=Bash, style=Shell, basicstyle=\tiny, breaklines=true]
@function conda_initialize {@
# !! Contents within this block are managed by 'conda init' !!
    __conda_setup="$('<prefix>/bin/conda' 'shell.bash' 'hook' 2> /dev/null)"
if [ $? -eq 0 ]; then
    eval "$__conda_setup"
else
if [ -f "<prefix>/etc/profile.d/conda.sh" ]; then
   . "<prefix>/etc/profile.d/conda.sh"
else
   export PATH="<prefix>/bin:$PATH"
fi
fi
unset __conda_setup

if [ -f "<prefix>/etc/profile.d/mamba.sh" ]; then 
    . "<prefix>/etc/profile.d/mamba.sh"
fi
# <<< conda initialize <<<
@}@
      \end{lstlisting}
    \end{column}
  \end{columns}
\end{frame}


%%%%%%%%%%%%%%%%%%%%%%%%%%%%%%%%%%%%%%%%%%%%%%%%%%%%%%%%%%%%%%%%%%%%%%%%%%%%%% 
\begin{frame}[fragile]
  \frametitle{Why this function in somebodies \texttt{.bashrc}?}
  \begin{docs}
  	\begin{itemize}[<+->]
  		\item \emph{every} time a user logs in, code in the \texttt{.bashrc} will be executed. Depending a conda setup, FS issues, etc. this can be \emph{slow}.
  		\item automatic inclusion of Conda/Mamba might cause interference with modules
  		\item Now, users can run \verb+conda_initialize+ in their login shell, jobs scripts, etc. upon demand and deactivate if needed.
  	\end{itemize}
  \end{docs}
\end{frame}

%%%%%%%%%%%%%%%%%%%%%%%%%%%%%%%%%%%%%%%%%%%%%%%%%%%%%%%%%%%%%%%%%%%%%%%%%%%%%%% 
\input{<++course.condarc_admins_file++>}

%%%%%%%%%%%%%%%%%%%%%%%%%%%%%%%%%%%%%%%%%%%%%%%%%%%%%%%%%%%%%%%%%%%%%%%%%%%%%%% 
\begin{frame}
  \frametitle{What we teach \ldots}
  \begin{columns}
  	\begin{column}{0.5\textwidth}
  		\begin{itemize}[<+->]
  			\item first we show them how to install software into en environment
  			\item later one environment per worfklow
  			\item then how to delegate the deployment to \Snakemake -- and live with a very basic environment
  		\end{itemize}
  	\end{column}
  	\begin{column}{0.5\textwidth}
  		\pause
  		\begin{hint}{MPI, Performance and the Rest}
  			%TODO wait for pixi
  			Currently, the basic deployment method is Conda. Working with Modules, e.\,g. for MPI-Software, or containers is \emph{an exception}.\newline
  			The community is working hard to mitigate performance issues.
  		\end{hint}
  	\end{column}
  \end{columns}
\end{frame}

