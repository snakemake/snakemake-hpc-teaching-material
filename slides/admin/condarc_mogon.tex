%%%%%%%%%%%%%%%%%%%%%%%%%%%%%%%%%%%%%%%%%%%%%%%%%%%%%%%%%%%%%%%%%%%%%%%%%%%%%%% 
\begin{frame}[fragile]
  \frametitle{Reducing Search Overhead - the \texttt{.condarc}-File}
  On many HPC clusters the number of Conda channels (the repositories) is reduced by the whitelisting. We recommend these resource settings - per user home:
  \begin{lstlisting}[language=Bash, style=Shell, basicstyle=\tiny]
$ cat ~/.condarc
create_default_packages:
  - setuptools # since Python is often needed
# these channels cover most of the required software
channels:
  - conda-forge
  - bioconda
  - defaults
  - r
proxy_servers: 
  http: http://webproxy.zdv.uni-mainz.de:8888
ssl_verify: false
auto_update_conda: false
always_yes: true # avoid confirmation(s)
env_prompt: '($(basename {default_env})) '
  \end{lstlisting}
  The last line ensures that storing an unnamed environment on Lustre, will not result in overly lenghty prompts.
\end{frame}
