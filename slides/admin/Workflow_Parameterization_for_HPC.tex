%%%%%%%%%%%%%%%%%%%%%%%%%%%%%%%%%%%%%%%%%%%%%%%%%%%%%%%%%%%%%%%%%%%%%%%%%%%%%%%%
\section{Global Settings for Workflow Parametrization}

%%%%%%%%%%%%%%%%%%%%%%%%%%%%%%%%%%%%%%%%%%%%%%%%%%%%%%%%%%%%%%%%%%%%%%%%%%%%%%%%
\begin{frame}
    \frametitle{Outline}
    \begin{columns}[t]
        \begin{column}{.5\textwidth}
            \tableofcontents[sections={1-7},currentsection]
        \end{column}
        \begin{column}{.5\textwidth}
            \tableofcontents[sections={8-15},currentsection]
        \end{column}
    \end{columns}
\end{frame}

%%%%%%%%%%%%%%%%%%%%%%%%%%%%%%%%%%%%%%%%%%%%%%%%%%%%%%%%%%%%%%%%%%%%%%%%%%%%%%%%
\begin{frame}
  \frametitle{What is this about?}
  \begin{question}[Questions]
   	\begin{itemize}
      \item Avoiding I/O contention! How?
      \item Accounting for FS Latency! How?
    \end{itemize}
  \end{question}
   \begin{docs}[Objectives]
   	 \begin{enumerate} 
        \item Learn how to tune \Snakemake{} to mitigate I/O contention.
        \item Learn how to configure \Snakemake{} to allow for file system latency.
    \end{enumerate}
  \end{docs}
\end{frame}

%%%%%%%%%%%%%%%%%%%%%%%%%%%%%%%%%%%%%%%%%%%%%%%%%%%%%%%%%%%%%%%%%%%%%%%%%%%%%%%%
\begin{frame}
	\frametitle{\Interlude{Random Access and File System Latency}}
	\begin{hint}
		Here, we explain to users what random access patterns are and how they arise.\newline
		We tell -- roughly -- the story of file system latencies, their causes and characteristics of different file systems and speeds.\newline
		\pause
		We do not feel it's appropriate to lecture you. Instead, we show how easy \Snakemake will offer a solution to I/O contention.
	\end{hint}
\end{frame}

%%%%%%%%%%%%%%%%%%%%%%%%%%%%%%%%%%%%%%%%%%%%%%%%%%%%%%%%%%%%%%%%%%%%%%%%%%%%%%%%
\subsection{Global Workflow Configuration}

%%%%%%%%%%%%%%%%%%%%%%%%%%%%%%%%%%%%%%%%%%%%%%%%%%%%%%%%%%%%%%%%%%%%%%%%%%%%%%%%
\begin{frame}[fragile]
  \frametitle{\Snakemake{} Profiles}
  \begin{hint}
  	Profiles can shorten command lines and can be an easy remedy for I/O issues!
  \end{hint}
  \pause
  Two kinds of profiles are supported:
  \begin{itemize}[<+->]
  	\item A global profile that is defined in a system-wide or user-specific configuration directory (on Linux, this will be \altverb{\~/.config/snakemake} or \altverb{/etc/xdg/snakemake}, you can find the answer for your system via \altverb{snakemake --help}).
  	\item A workflow specific profile that is defined via a flag (\altverb{--workflow-profile}) or searched in a default location (profile/default) in the working directory or next to the \altverb{Snakefile}.
  \end{itemize}
\end{frame}

%%%%%%%%%%%%%%%%%%%%%%%%%%%%%%%%%%%%%%%%%%%%%%%%%%%%%%%%%%%%%%%%%%%%%%%%%%%%%%%%
\begin{frame}[fragile]
  \frametitle{\Snakemake{} Profile - the File}
    \begin{onlyenv}<1| handout:0>
      Our first line defines the so-called executor to be set for SLURM:
      \begin{lstlisting}[language=Bash, style=Shell]
executor: slurm
     \end{lstlisting}
     No more, \altverb{snakemake --executor slurm ...}!
   \end{onlyenv}
   \begin{onlyenv}<2| handout:0>
   	 The next line tells Snakemake to wait for a few seconds, if output files are not present. This is more than enough time, even for NFS-Filesystems (usually). 
   	 \begin{lstlisting}[language=Bash, style=Shell]
executor: slurm
latency-wait: 5
   	 \end{lstlisting}
    \end{onlyenv}
    \begin{onlyenv}<3| handout:0>
      The entire rest, will tell the storage plugin (\altverb{snakemake-storage-plugin-fs}) to stage in to the node-local storage on Mogon, for \emph{every} job and to copy back your results. When dealing with I/O intensive jobs, this can boost your performance tremendously. 
      \begin{lstlisting}[language=Bash, style=Shell]
executor: slurm
latency-wait: 5
default-storage-provider: fs
shared-fs-usage:
  - persistence
  - sources
  - source-cache
remote-job-local-storage-prefix: /localscratch/$SLURM_JOB_ID
local-storage-prefix: /dev/shm/$USER
    \end{lstlisting}
   \end{onlyenv}
  \begin{onlyenv}<4| handout:0>
  	%TODO remove this part, once the snakemake release is ready
  	\begin{warning}
  	  Currently, we are working on way to annotate susceptible I/O pattern, the highlighted parts are not yet functional.
  	\end{warning}
  	%You may copy this setup from \texttt{<++course.pathtosetup++>/config.yaml} to the \altverb{\~/.config/snakemake}-folder - unless your local admins provide a cluster-wide configuration.
  	\begin{lstlisting}[language=Bash, style=Shell]
executor: slurm
@latency-wait: 5@
@default-storage-provider: fs@
@shared-fs-usage:@
@  - persistence@
@  - sources@
@  - source-cache@
remote-job-local-storage-prefix: /localscratch/$SLURM_JOB_ID 
local-storage-prefix: /dev/shm/$USER 
     \end{lstlisting}
  \end{onlyenv}
   \begin{onlyenv}<5| handout:1>
   	The complete configuration out to be in \altverb{\~/.config/snakemake/config.yaml} per user or globally on your cluster at \altverb{/etc/xdg/snakemake}.
   	\begin{lstlisting}[language=Bash, style=Shell]
executor: slurm
latency-wait: 5
default-storage-provider: fs
shared-fs-usage:
- persistence
- sources
- source-cache
remote-job-local-storage-prefix: /localscratch/$SLURM_JOB_ID
local-storage-prefix: /dev/shm/$USER
    \end{lstlisting}
  \end{onlyenv}
\end{frame}

