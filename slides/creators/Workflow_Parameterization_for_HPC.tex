%%%%%%%%%%%%%%%%%%%%%%%%%%%%%%%%%%%%%%%%%%%%%%%%%%%%%%%%%%%%%%%%%%%%%%%%%%%%%%%%
\section{Parametizing your Workflow - II}

%%%%%%%%%%%%%%%%%%%%%%%%%%%%%%%%%%%%%%%%%%%%%%%%%%%%%%%%%%%%%%%%%%%%%%%%%%%%%%%%
\begin{frame}
    \frametitle{Outline}
    \begin{columns}[t]
        \begin{column}{.5\textwidth}
            \tableofcontents[sections={1-9},currentsection]
        \end{column}
        \begin{column}{.5\textwidth}
            \tableofcontents[sections={10-18},currentsection]
        \end{column}
    \end{columns}
\end{frame}

%%%%%%%%%%%%%%%%%%%%%%%%%%%%%%%%%%%%%%%%%%%%%%%%%%%%%%%%%%%%%%%%%%%%%%%%%%%%%%%%
\begin{frame}
  \frametitle{What is this about?}
  \begin{question}[Questions]
   	\begin{itemize}
      \item How do we avoid I/O contention?
      \item How do we account for file system latency?
    \end{itemize}
  \end{question}
   \begin{docs}[Objectives]
   	 \begin{enumerate} 
        \item Learn how to tune \Snakemake{} to mitigate I/O contention.
        \item Learn how to configure \Snakemake{} to allow for file system latency.
    \end{enumerate}
  \end{docs}
\end{frame}

%%%%%%%%%%%%%%%%%%%%%%%%%%%%%%%%%%%%%%%%%%%%%%%%%%%%%%%%%%%%%%%%%%%%%%%%%%%%%%%%
\begin{frame}
	\frametitle{\Interlude{What is Random Access?}}
	\vspace{-1.5em}
	\begin{figure}
		\centering
		\begin{tikzpicture}
			\node[name=sa] at (3.5,10) { \lhref{https://en.wikipedia.org/wiki/Random_access}{Sequential Access} };
			\foreach[count=\y from 2] \x in {1,...,11}{
				\draw[thick,black,midway,draw] (\x,8.75) rectangle node[name=s\x] {} (\y,8.25);
			}       
			\foreach[count=\y from 2] \x in  {1, ..., 10} {
					\path[->] ([yshift=.6em]s\x.north west) edge[bend left=30] ([yshift=.6em]s\y.north west);
			}
			%         \path[->] ([yshift=.6em]s1.north) edge [bend left=30] ([yshift=.6em]s2.north) ;
			
			\node[name=ra] at (3.5,7) { Random Access };
			\foreach[count=\y from 2] \x in {1,...,11}{
				\draw[thick,black,midway,draw] (\x,5.75) rectangle node[name=r\x] {} (\y,5.25);
			}
		
			\path[->] ([yshift=.6em]r1.north) edge [bend left=30] ([yshift=.6em]r5.north) ;
			\path[->] ([yshift=.6em]r5.north) edge [bend right=60] ([yshift=.6em]r2.north) ;
			\path[->] ([yshift=.6em]r2.north) edge [bend left=30] ([yshift=.6em]r3.north) ;
			\path[->] ([yshift=.6em]r3.north) edge [bend left=30] ([yshift=.6em]r11.north) ;
			\path[->] ([yshift=.6em]r11.north) edge [bend right=30] ([yshift=.6em]r7.north) ;
			\path[->] ([yshift=.6em]r7.north) edge [bend right=30] ([yshift=.6em]r6.north) ;
			\path[->] ([yshift=.6em]r6.north) edge [bend left=60] ([yshift=.6em]r8.north) ;
			\path[->] ([yshift=.6em]r8.north) edge [bend right=50] ([yshift=.6em]r4.north) ;
		\end{tikzpicture}
	\end{figure}
\end{frame}

%%%%%%%%%%%%%%%%%%%%%%%%%%%%%%%%%%%%%%%%%%%%%%%%%%%%%%%%%%%%%%%%%%%%%%%%%%%%%%%%
\begin{frame}
  \frametitle{\Interlude{What is File System Latency?}}
    \centering
	\begin{tikzpicture}[line cap=rect,line width=3pt, 
		datastore/.style={draw, rounded rectangle, rounded rectangle east arc=concave, rounded rectangle arc length=150},
		]
	  \tikzstyle{storage} = [rectangle, minimum width=3cm, minimum height=1cm, text width=3cm, text centered, draw=black]
	  
	  \filldraw [fill=cyan] (0,0) circle [radius=1cm];
	  \foreach \angle [count=\xi] in {60,30,...,-270}
	  {
		\draw[line width=0.5pt] (\angle:0.9cm) -- (\angle:1cm);
		\node[font=\small] at (\angle:0.68cm) {\textsf{\xi}};
	  }
	  \foreach \angle in {0,90,180,270}
	  \draw[line width=1pt] (\angle:0.8cm) -- (\angle:1cm);
	  \draw (0,0) -- (120:0.4cm);
	  \draw (0,0) -- (90:.5cm);
	  
	  \node (sto1) [datastore] at (-4, 0) {Storage};
	  \node at (4, 0) {\includegraphics[width=.25\textwidth]{misc/data_center.png}};
  \end{tikzpicture}
  \begin{docs}{File System Latency}
  	The time it takes from the file system to the client and back.
  \end{docs}
\end{frame}

%%%%%%%%%%%%%%%%%%%%%%%%%%%%%%%%%%%%%%%%%%%%%%%%%%%%%%%%%%%%%%%%%%%%%%%%%%%%%%%%
\subsection{Global Workflow Configuration}

%%%%%%%%%%%%%%%%%%%%%%%%%%%%%%%%%%%%%%%%%%%%%%%%%%%%%%%%%%%%%%%%%%%%%%%%%%%%%%%%
\begin{frame}[fragile]
	\frametitle{The \texttt{Snakemake} \texttt{resources} Section}
	\texttt{Snakemake} rules provide an additional \altverb{resource} section:
	\begin{lstlisting}[language=Python,style=Python]
		rule <name>:
		...
		resources:
		partition='parallel',
		mem_mb=1800,
		cpus_per_task=4
	\end{lstlisting}
	\begin{hint}
		Note the \textbf{,}!
	\end{hint}
	\pause
	\begin{docs}
		You \emph{may} define \emph{any} resource keyword within any rule.
	\end{docs}
\end{frame}
