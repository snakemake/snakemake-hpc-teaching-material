%%%%%%%%%%%%%%%%%%%%%%%%%%%%%%%%%%%%%%%%%%%%%%%%%%%%%%%%%%%%%%%%%%%%%%%%%%%%%%%%
\section{Parametizing your Workflow - II}

%%%%%%%%%%%%%%%%%%%%%%%%%%%%%%%%%%%%%%%%%%%%%%%%%%%%%%%%%%%%%%%%%%%%%%%%%%%%%%%%
\begin{frame}
    \frametitle{Outline}
    \begin{columns}[t]
        \begin{column}{.5\textwidth}
            \tableofcontents[sections={1-9},currentsection]
        \end{column}
        \begin{column}{.5\textwidth}
            \tableofcontents[sections={10-18},currentsection]
        \end{column}
    \end{columns}
\end{frame}

%%%%%%%%%%%%%%%%%%%%%%%%%%%%%%%%%%%%%%%%%%%%%%%%%%%%%%%%%%%%%%%%%%%%%%%%%%%%%%%%
\begin{frame}
  \frametitle{What is this about?}
  \begin{question}[Questions]
   	\begin{itemize}
      \item How do we avoid I/O contention?
      \item How do we account for file system latency?
    \end{itemize}
  \end{question}
   \begin{docs}[Objectives]
   	 \begin{enumerate} 
        \item Learn how to tune \Snakemake{} to mitigate I/O contention.
        \item Learn how to configure \Snakemake{} to allow for file system latency.
    \end{enumerate}
  \end{docs}
\end{frame}

%%%%%%%%%%%%%%%%%%%%%%%%%%%%%%%%%%%%%%%%%%%%%%%%%%%%%%%%%%%%%%%%%%%%%%%%%%%%%%%%
\begin{frame}
	\frametitle{\Interlude{What is Random Access?}}
	\vspace{-1.5em}
	\begin{figure}
		\centering
		\begin{tikzpicture}
			\node[name=sa] at (3.5,10) { \lhref{https://en.wikipedia.org/wiki/Random_access}{Sequential Access} };
			\foreach[count=\y from 2] \x in {1,...,11}{
				\draw[thick,black,midway,draw] (\x,8.75) rectangle node[name=s\x] {} (\y,8.25);
			}       
			\foreach[count=\y from 2] \x in  {1, ..., 10} {
					\path[->] ([yshift=.6em]s\x.north west) edge[bend left=30] ([yshift=.6em]s\y.north west);
			}
			%         \path[->] ([yshift=.6em]s1.north) edge [bend left=30] ([yshift=.6em]s2.north) ;
			
			\node[name=ra] at (3.5,7) { Random Access };
			\foreach[count=\y from 2] \x in {1,...,11}{
				\draw[thick,black,midway,draw] (\x,5.75) rectangle node[name=r\x] {} (\y,5.25);
			}
		
			\path[->] ([yshift=.6em]r1.north) edge [bend left=30] ([yshift=.6em]r5.north) ;
			\path[->] ([yshift=.6em]r5.north) edge [bend right=60] ([yshift=.6em]r2.north) ;
			\path[->] ([yshift=.6em]r2.north) edge [bend left=30] ([yshift=.6em]r3.north) ;
			\path[->] ([yshift=.6em]r3.north) edge [bend left=30] ([yshift=.6em]r11.north) ;
			\path[->] ([yshift=.6em]r11.north) edge [bend right=30] ([yshift=.6em]r7.north) ;
			\path[->] ([yshift=.6em]r7.north) edge [bend right=30] ([yshift=.6em]r6.north) ;
			\path[->] ([yshift=.6em]r6.north) edge [bend left=60] ([yshift=.6em]r8.north) ;
			\path[->] ([yshift=.6em]r8.north) edge [bend right=50] ([yshift=.6em]r4.north) ;
		\end{tikzpicture}
	\end{figure}
\end{frame}

%%%%%%%%%%%%%%%%%%%%%%%%%%%%%%%%%%%%%%%%%%%%%%%%%%%%%%%%%%%%%%%%%%%%%%%%%%%%%%%%
\begin{frame}
	\frametitle{\Interlude{What is Random Access?}}
	\begin{question}
	  \begin{itemize}
	  	\item What causes Random Access?
	  	\item Why is it harmful? What can we do?
	  \end{itemize}
    \end{question}
	\pause
	\begin{columns}[t]
	  \begin{column}{0.5\textwidth}
	  	Causes:
	  	\hrule
	  	\begin{itemize}
	  	  \item a number of (threaded) apps accessing the same file space (e.g. reference data)
	  	  \item a number of apps accessing a file space exceeding the file system cache size
	  	\end{itemize}
  	     Will slow parallel file systems and your data analysis!
	  \end{column}
      \begin{column}{0.5\textwidth}
      	Remedies:
      	\hrule
      	\begin{itemize}
      		\item copy data to/from compute nodes equipped with SSD
      		\item use a RAM disk (RAM = random access memory) - which many clusters provide
      	\end{itemize}
      \end{column}
	\end{columns}	
\end{frame}

%%%%%%%%%%%%%%%%%%%%%%%%%%%%%%%%%%%%%%%%%%%%%%%%%%%%%%%%%%%%%%%%%%%%%%%%%%%%%%%%
\begin{frame}
  \frametitle{\Interlude{What is File System Latency?}}
    \centering
	\begin{tikzpicture}[line cap=rect,line width=3pt, 
		datastore/.style={draw, rounded rectangle, rounded rectangle east arc=concave, rounded rectangle arc length=150},
		]
	  \tikzstyle{storage} = [rectangle, minimum width=3cm, minimum height=1cm, text width=3cm, text centered, draw=black]
	  
	  \filldraw [fill=cyan] (0,0) circle [radius=1cm];
	  \foreach \angle [count=\xi] in {60,30,...,-270}
	  {
		\draw[line width=0.5pt] (\angle:0.9cm) -- (\angle:1cm);
		\node[font=\small] at (\angle:0.68cm) {\textsf{\xi}};
	  }
	  \foreach \angle in {0,90,180,270}
	  \draw[line width=1pt] (\angle:0.8cm) -- (\angle:1cm);
	  \draw (0,0) -- (120:0.4cm);
	  \draw (0,0) -- (90:.5cm);
	  
	  \node (sto1) [datastore] at (-4, 0) {Storage};
	  \node at (4, 0) {\includegraphics[width=.25\textwidth]{misc/data_center.png}};
  \end{tikzpicture}
  \begin{docs}{File System Latency}
  	The time it takes from the file system to the client and back.
  \end{docs}
\end{frame}

%%%%%%%%%%%%%%%%%%%%%%%%%%%%%%%%%%%%%%%%%%%%%%%%%%%%%%%%%%%%%%%%%%%%%%%%%%%%%%%%
\subsection{Global Workflow Configuration}

%%%%%%%%%%%%%%%%%%%%%%%%%%%%%%%%%%%%%%%%%%%%%%%%%%%%%%%%%%%%%%%%%%%%%%%%%%%%%%%%
\begin{frame}[fragile]
  \frametitle{\Snakemake{} Profiles}
  Two kinds of profiles are supported:
  \begin{itemize}[<+->]
  	\item A global profile that is defined in a system-wide or user-specific configuration directory (on Linux, this will be \altverb{\~/.config/snakemake} and \altverb{/etc/xdg/snakemake}, you can find the answer for your system via \altverb{snakemake --help}).
  	\item A workflow specific profile that is defined via a flag (\altverb{--workflow-profile}) or searched in a default location (profile/default) in the working directory or next to the \altverb{Snakefile}.
  \end{itemize}
\end{frame}

%%%%%%%%%%%%%%%%%%%%%%%%%%%%%%%%%%%%%%%%%%%%%%%%%%%%%%%%%%%%%%%%%%%%%%%%%%%%%%%%
\begin{frame}[fragile]
  \frametitle{Your Profile}
    \begin{onlyenv}<1| handout:0>
      Our first line forces the use of the \lhref{https://yte-template-engine.github.io/}{\bf{Y}AML \bf{T}emplate \bf{E}ngine} to parse the following lines accordingly.
      \begin{lstlisting}[language=Bash, style=Shell]
__use_yte__: true  	 
      \end{lstlisting}
    \end{onlyenv}
    \begin{onlyenv}<2| handout:0>
      Now, our default executor is SLURM.
      \begin{lstlisting}[language=Bash, style=Shell]
__use_yte__: true
executor: slurm
     \end{lstlisting}
     No more, \altverb{snakemake --executor slurm ...}!
   \end{onlyenv}
   \begin{onlyenv}<3| handout:0>
   	 The next line tells Snakemake to wait for a minute, if output files are not present. This is more than enough time, even for NFS-Filesystems. 
   	 \begin{lstlisting}[language=Bash, style=Shell]
__use_yte__: true
executor: slurm
latency-wait: 60
   	 \end{lstlisting}
    \end{onlyenv}
    \begin{onlyenv}<4| handout:0>
      The entire rest, will tell the storage plugin (\altverb{snakemake-storage-plugin-fs}) to stage in to the node-local storage on Mogon, for \emph{every} job and to copy back your results. When dealing with I/O intensive jobs, this can boost your performance tremendously. 
      \begin{lstlisting}[language=Bash, style=Shell]
__use_yte__: true
executor: slurm
latency-wait: 60
default-storage-provider: fs
shared-fs-usage:
  - persistence
  - sources
  - source-cache
local-storage-prefix: /localscratch/$SLURM_JOB_ID
    \end{lstlisting}
   \end{onlyenv}
  \begin{onlyenv}<5| handout:0>
  	%TODO remove this part, once the snakemake release is ready
  	\begin{warning}
  	  Currently, we are working on way to annotate susceptible I/O pattern, the highlighted parts are not yet functional.
  	\end{warning}
  	%You may copy this setup from \texttt{\configparam{pathtosetup}/config.yaml} to the \altverb{\~/.config/snakemake}-folder - unless your local admins provide a cluster-wide configuration.
  	\begin{lstlisting}[language=Bash, style=Shell]
__use_yte__: true
executor: slurm
@latency-wait: 60@
@default-storage-provider: fs@
@shared-fs-usage:@
@  - persistence@
@  - sources@
@  - source-cache@
@local-storage-prefix: /localscratch/$SLURM_JOB_ID@
  	 \end{lstlisting}
  \end{onlyenv}
   \begin{onlyenv}<6| handout:1>
   	The complete configuration out to be in \altverb{\~/.config/snakemake/config.yaml}
   	\begin{lstlisting}[language=Bash, style=Shell]
__use_yte__: true
executor: slurm
latency-wait: 60
default-storage-provider: fs
shared-fs-usage:
- persistence
- sources
- source-cache
local-storage-prefix: /localscratch/$SLURM_JOB_ID
    \end{lstlisting}
  \end{onlyenv}
\end{frame}


%%%%%%%%%%%%%%%%%%%%%%%%%%%%%%%%%%%%%%%%%%%%%%%%%%%%%%%%%%%%%%%%%%%%%%%%%%%%%%%%
\begin{frame}[fragile]
	\frametitle{\HandsOn{Running with a Profile I}}
	Enter 
	\begin{lstlisting}[language=Bash, style=Shell]
export SNAKEMAKE_PROFILE="$HOME/.config/snakemake"
    \end{lstlisting}
	in your \altverb{.bashrc}
\end{frame}

%%%%%%%%%%%%%%%%%%%%%%%%%%%%%%%%%%%%%%%%%%%%%%%%%%%%%%%%%%%%%%%%%%%%%%%%%%%%%%%%
\begin{frame}[fragile]
	\frametitle{\HandsOn{Running with a Profile II}}
	Copy the configuration file:
	\begin{lstlisting}[language=Bash, style=Shell]
$ mkdir -p ~/.config/snakemake
$ cp ~/workflows/config.yaml \
> ~/.config/snakemake/.
	\end{lstlisting}
    Activate the new settings:
    \begin{lstlisting}[language=Bash, style=Shell]
$ source ~/.bashrc
    \end{lstlisting}
    And run the workflow - just for fun:
    \begin{lstlisting}[language=Bash, style=Shell]
$ snakemake -F # just this one flag!
    \end{lstlisting}
\end{frame}

	

